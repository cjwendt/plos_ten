% Template for PLoS
% Version 3.5 March 2018
%
% % % % % % % % % % % % % % % % % % % % % %
%
% -- IMPORTANT NOTE
%
% This template contains comments intended
% to minimize problems and delays during our production
% process. Please follow the template instructions
% whenever possible.
%
% % % % % % % % % % % % % % % % % % % % % % %
%
% Once your paper is accepted for publication,
% PLEASE REMOVE ALL TRACKED CHANGES in this file
% and leave only the final text of your manuscript.
% PLOS recommends the use of latexdiff to track changes during review, as this will help to maintain a clean tex file.
% Visit https://www.ctan.org/pkg/latexdiff?lang=en for info or contact us at latex@plos.org.
%
%
% There are no restrictions on package use within the LaTeX files except that
% no packages listed in the template may be deleted.
%
% Please do not include colors or graphics in the text.
%
% The manuscript LaTeX source should be contained within a single file (do not use \input, \externaldocument, or similar commands).
%
% % % % % % % % % % % % % % % % % % % % % % %
%
% -- FIGURES AND TABLES
%
% Please include tables/figure captions directly after the paragraph where they are first cited in the text.
%
% DO NOT INCLUDE GRAPHICS IN YOUR MANUSCRIPT
% - Figures should be uploaded separately from your manuscript file.
% - Figures generated using LaTeX should be extracted and removed from the PDF before submission.
% - Figures containing multiple panels/subfigures must be combined into one image file before submission.
% For figure citations, please use "Fig" instead of "Figure".
% See http://journals.plos.org/plosone/s/figures for PLOS figure guidelines.
%
% Tables should be cell-based and may not contain:
% - spacing/line breaks within cells to alter layout or alignment
% - do not nest tabular environments (no tabular environments within tabular environments)
% - no graphics or colored text (cell background color/shading OK)
% See http://journals.plos.org/plosone/s/tables for table guidelines.
%
% For tables that exceed the width of the text column, use the adjustwidth environment as illustrated in the example table in text below.
%
% % % % % % % % % % % % % % % % % % % % % % % %
%
% -- EQUATIONS, MATH SYMBOLS, SUBSCRIPTS, AND SUPERSCRIPTS
%
% IMPORTANT
% Below are a few tips to help format your equations and other special characters according to our specifications. For more tips to help reduce the possibility of formatting errors during conversion, please see our LaTeX guidelines at http://journals.plos.org/plosone/s/latex
%
% For inline equations, please be sure to include all portions of an equation in the math environment.
%
% Do not include text that is not math in the math environment.
%
% Please add line breaks to long display equations when possible in order to fit size of the column.
%
% For inline equations, please do not include punctuation (commas, etc) within the math environment unless this is part of the equation.
%
% When adding superscript or subscripts outside of brackets/braces, please group using {}.
%
% Do not use \cal for caligraphic font.  Instead, use \mathcal{}
%
% % % % % % % % % % % % % % % % % % % % % % % %
%
% Please contact latex@plos.org with any questions.
%
% % % % % % % % % % % % % % % % % % % % % % % %

\documentclass[10pt,letterpaper]{article}
\usepackage[top=0.85in,left=2.75in,footskip=0.75in]{geometry}

% amsmath and amssymb packages, useful for mathematical formulas and symbols
\usepackage{amsmath,amssymb}

% Use adjustwidth environment to exceed column width (see example table in text)
\usepackage{changepage}

% Use Unicode characters when possible
\usepackage[utf8x]{inputenc}

% textcomp package and marvosym package for additional characters
\usepackage{textcomp,marvosym}

% cite package, to clean up citations in the main text. Do not remove.
% \usepackage{cite}

% Use nameref to cite supporting information files (see Supporting Information section for more info)
\usepackage{nameref,hyperref}

% line numbers
\usepackage[right]{lineno}

% ligatures disabled
\usepackage{microtype}
\DisableLigatures[f]{encoding = *, family = * }

% color can be used to apply background shading to table cells only
\usepackage[table]{xcolor}

% array package and thick rules for tables
\usepackage{array}

% create "+" rule type for thick vertical lines
\newcolumntype{+}{!{\vrule width 2pt}}

% create \thickcline for thick horizontal lines of variable length
\newlength\savedwidth
\newcommand\thickcline[1]{%
  \noalign{\global\savedwidth\arrayrulewidth\global\arrayrulewidth 2pt}%
  \cline{#1}%
  \noalign{\vskip\arrayrulewidth}%
  \noalign{\global\arrayrulewidth\savedwidth}%
}

% \thickhline command for thick horizontal lines that span the table
\newcommand\thickhline{\noalign{\global\savedwidth\arrayrulewidth\global\arrayrulewidth 2pt}%
\hline
\noalign{\global\arrayrulewidth\savedwidth}}


% Remove comment for double spacing
%\usepackage{setspace}
%\doublespacing

% Text layout
\raggedright
\setlength{\parindent}{0.5cm}
\textwidth 5.25in
\textheight 8.75in

% Bold the 'Figure #' in the caption and separate it from the title/caption with a period
% Captions will be left justified
\usepackage[aboveskip=1pt,labelfont=bf,labelsep=period,justification=raggedright,singlelinecheck=off]{caption}
\renewcommand{\figurename}{Fig}

% Use the PLoS provided BiBTeX style
% \bibliographystyle{plos2015}

% Remove brackets from numbering in List of References
\makeatletter
\renewcommand{\@biblabel}[1]{\quad#1.}
\makeatother



% Header and Footer with logo
\usepackage{lastpage,fancyhdr,graphicx}
\usepackage{epstopdf}
%\pagestyle{myheadings}
\pagestyle{fancy}
\fancyhf{}
%\setlength{\headheight}{27.023pt}
%\lhead{\includegraphics[width=2.0in]{PLOS-submission.eps}}
\rfoot{\thepage/\pageref{LastPage}}
\renewcommand{\headrulewidth}{0pt}
\renewcommand{\footrule}{\hrule height 2pt \vspace{2mm}}
\fancyheadoffset[L]{2.25in}
\fancyfootoffset[L]{2.25in}
\lfoot{\today}

%% Include all macros below

\newcommand{\lorem}{{\bf LOREM}}
\newcommand{\ipsum}{{\bf IPSUM}}






\usepackage{forarray}
\usepackage{xstring}
\newcommand{\getIndex}[2]{
  \ForEach{,}{\IfEq{#1}{\thislevelitem}{\number\thislevelcount\ExitForEach}{}}{#2}
}

\setcounter{secnumdepth}{0}

\newcommand{\getAff}[1]{
  \getIndex{#1}{Stats,Math,ERHS}
}

\providecommand{\tightlist}{%
  \setlength{\itemsep}{0pt}\setlength{\parskip}{0pt}}

\begin{document}
\vspace*{0.2in}

% Title must be 250 characters or less.
\begin{flushleft}
{\Large
\textbf\newline{Ten simple rules for selecting an R package} % Please use "sentence case" for title and headings (capitalize only the first word in a title (or heading), the first word in a subtitle (or subheading), and any proper nouns).
}
\newline
% Insert author names, affiliations and corresponding author email (do not include titles, positions, or degrees).
\\
Caroline J. Wendt\textsuperscript{\getAff{Stats}, \getAff{Math}},
G. Brooke Anderson\textsuperscript{\getAff{ERHS}}\textsuperscript{*}\\
\bigskip
\textbf{\getAff{Stats}}Department of Statistics, Colorado State University, Fort Collins,
Colorado, United States of America\\
\textbf{\getAff{Math}}Department of Mathematics, Colorado State University, Fort Collins,
Colorado, United States of America\\
\textbf{\getAff{ERHS}}Department of Environmental \& Radiological Health Sciences, Colorado
State University, Fort Collins, Colorado, United States of America\\
\bigskip
* Corresponding author: Brooke.Anderson@colostate.edu\\
\end{flushleft}
% Please keep the abstract below 300 words
\section*{Abstract}
R is an increasingly preferred software environment for data analytics
and statistical computing among scientists and practitioners. Packages
markedly extend R's utility and ameliorate inefficient solutions. We
outline ten simple rules for finding relevant packages and determining
which is optimal for your desired use.

% Please keep the Author Summary between 150 and 200 words
% Use first person. PLOS ONE authors please skip this step.
% Author Summary not valid for PLOS ONE submissions.
\section*{Author summary}
Write the author summary here. Do we want to include and author summary?

\linenumbers

% Use "Eq" instead of "Equation" for equation citations.
\emph{Text based on plos sample manuscript, see
\url{http://journals.plos.org/ploscompbiol/s/latex}}

\hypertarget{disclaimer}{%
\section{Disclaimer?}\label{disclaimer}}

Do we need to include a disclaimer in the margin like the one from
{[}1{]} that states: ``\textbf{Competing Interests}: The authors have no
affiliation with GitHub, nor with any other commercial entity mentioned
in this article. The views described here reflect their own views
without input from any third party organization.''

\begin{itemize}
\tightlist
\item
  RStudio
\item
  ROpenSci
\end{itemize}

\hypertarget{funding-acknowledgment}{%
\section{Funding acknowledgment?}\label{funding-acknowledgment}}

Do we need to include a funding acknowledgment in the margin as in the
examples?

\hypertarget{introduction}{%
\section{Introduction}\label{introduction}}

R is a language and environment for statistical computing and graphics
that was developed by statisticians and is collaboratively maintained by
an international core group of contributors. Unlike many popular
proprietary languages (e.g., MATLAB, SAS, SPSS), R is highly extensible,
free and open-source software; the user can access and thus change,
extend, and share code for desired applications. Accordingly, a vibrant
community of R users has emerged, many of which engage in the
development of extensions to the functionality of base R software known
as packages. There are many analogies in computing that draw comparisons
between programming and culinary arts: recipe structures, coding
cookbooks, and the like. To conceptualize packages, imagine you are the
chef, R is the kitchen, and packages are the special gadgets which allow
you to cook and bake new recipes. R packages are coding delectables that
enable the user to perform practical tasks and solve problems with
interesting techniques.

Are there R packages for wrangling and cleaning data frames, designing
interactive apps for data visualization, or performing dimensionality
reduction? Yes! How do you find an R package that will help you train
regression and classification models, assess the beta diversity of a
population, or analyze gene expression microarray data? The answer is
not as simple; there are tens of thousands of R packages. As a natural
consequence of the open-source nature of R, there is variation in the
quality of different packages among the numerous choices that exist. The
advanced R user---having developed an intuition for their workflow---may
tend to be relatively confident when searching for and selecting
packages. By contrast, a common experience that characterizes learning R
at the outset is the struggle to 1) find a package to accomplish a
particular task or solve a problem of interest and 2) choose the best
package to perform that task. Even so, there remain obscure and
complicated problems that morph selecting an R package into a barrier
despite experience.

In coding as in life, we endeavor to make choices that optimize
outcomes. Just as one may go about shopping for shoes, deciding which
graduate program to pursue, or conducting a literature review, there is
a science behind selection. We inform our decisions by assessing,
comparing, and filtering options based on indicators of quality such as
utility, association, and reputation. Likewise, choosing an R package
requires attending to similar details. We outline ten simple rules for
finding and selecting R packages so that you will spend less time
searching for the right tools and more time coding delicious recipes.

\hypertarget{rule-1-consider-your-purpose}{%
\section{Rule 1: Consider your
purpose}\label{rule-1-consider-your-purpose}}

There are often several different ways to accomplish a task or arrive at
a solution while programming, albeit some ways are more elegant and
efficient than others. While there are certainly ways to cook up an
algorithm using for loops and conditionals in base R, a relevant package
may accomplish the same goal in a more reproducible manner with less
code and fewer bugs. If you define your purpose by making observations
about and considering limitations of your current toolbox before you
start searching for new tools, you will be more likely to recognize what
you do (and don't) need.

Think about what you are doing (or trying to do) and how you might like
to do it. Identify the type of inputs you have and envision working with
them; contemplate the desired outputs and corresponding format. Which
existing functionalities in base R could be improved in context of your
problem? Which new functionalities would you like to add to base R to
expand what you can do? You may have a unique or particular task that
requires highly specialized functions. Some R packages are simple and
have a very specific use. For example, there is a package for converting
English letters to numbers as on a telephone keypad {[}2{]}. On the
other hand, you may be faced with an extensive task that justifies a
broader framework with several functions that form a cohesive package.
Numerous processes that involve data---although varying in
application---are ubiquitous. Data manipulation is one such common task
that has been streamlined by packages such as \texttt{dplyr} and
\texttt{data.table} {[}3,4{]}.

In general, the more reasonable it is for your given task to be
abstracted away from its context, the more plausible it is that someone
has generalized its themes, bundled them up, and branded them in a nice
way---often \emph{much} nicer than expected. Nevertheless, there are
indeed packages for seemingly singular tasks, which you may be
pleasantly surprised to discover. You need not invent the wheel before
checking to see if it already exists, but before you sift through
options, having some notion of what you need will narrow your search.

\hypertarget{rule-2-spend-time-searching-find-and-collect-options}{%
\section{Rule 2: Spend time searching; find and collect
options}\label{rule-2-spend-time-searching-find-and-collect-options}}

\hypertarget{rule-3-check-how-its-shared}{%
\section{Rule 3: Check how it's
shared}\label{rule-3-check-how-its-shared}}

\hypertarget{rule-4-explore-the-availability-and-quality-of-help}{%
\section{Rule 4: Explore the availability and quality of
help}\label{rule-4-explore-the-availability-and-quality-of-help}}

\hypertarget{rule-5-verify-the-credibility-of-the-authors}{%
\section{Rule 5: Verify the credibility of the
author(s)}\label{rule-5-verify-the-credibility-of-the-authors}}

\hypertarget{rule-6-investigate-the-package-development}{%
\section{Rule 6: Investigate the package
development}\label{rule-6-investigate-the-package-development}}

\hypertarget{rule-7-read-research-literature-seek-evidence-of-peer-review}{%
\section{Rule 7: Read, research literature, seek evidence of peer
review}\label{rule-7-read-research-literature-seek-evidence-of-peer-review}}

\hypertarget{rule-8-quantify-how-established-the-package-is}{%
\section{Rule 8: Quantify how established the package
is}\label{rule-8-quantify-how-established-the-package-is}}

\hypertarget{rule-9-put-the-package-to-the-test}{%
\section{Rule 9: Put the package to the
test}\label{rule-9-put-the-package-to-the-test}}

\hypertarget{rule-10-develop-your-own-package}{%
\section{Rule 10: Develop your own
package}\label{rule-10-develop-your-own-package}}

\hypertarget{conclusion}{%
\section{Conclusion}\label{conclusion}}

Computational reproducibility is surfacing as a central axiom in
academia as researchers identify the need for means by which they can
implement transparent systems. As a corollary, former approaches are
often found to be at odds with productivity and collaboration. The R
programming language is increasingly used by researchers in
computational biology and bioinformatics, a discipline among many that
is generating extensive heterogenous and complex data that demands
sophisticated tools and rigorous methods {[}5,6{]}.

R packages are a defining feature of the language insofar as many are
robust and easily learnable. Packages greatly enhance the user
experience and enable you to be more efficient and effective at learning
from data regardless of prior experience. However, the sheer quantity
and potential complexity of available R packages can undermine their
collective benefits. Finding and choosing packages, particularly for
beginners, can be daunting and convoluted. R users often struggle to
sift through the tools at their disposal and wonder how to distinguish
appropriate usage. These ten simple rules for navigating the shared code
in the R community are intended to serve as a valuable page in your
computing cookbook---one that will evolve into intuition yet remain a
reliable reference. May searching for and selecting proper tools no
longer spoil your appetite and dissuade you from discovering, trying,
creating, and sharing new recipes.

\hypertarget{supporting-information}{%
\section{Supporting information}\label{supporting-information}}

Do we need to include any supporting information?

\hypertarget{acknowledgements}{%
\section{Acknowledgements}\label{acknowledgements}}

Do we need to include any acknowledgements?

\newpage

\hypertarget{list-of-10-rules}{%
\section{List of 10 rules}\label{list-of-10-rules}}

\hypertarget{currently-in-no-particular-order-and-not-precisely-worded}{%
\subsection{(currently in no particular order and not precisely
worded)}\label{currently-in-no-particular-order-and-not-precisely-worded}}

\begin{enumerate}
\def\labelenumi{\arabic{enumi}.}
\tightlist
\item
  \textbf{Consider your purpose}
\end{enumerate}

\begin{itemize}
\tightlist
\item
  features
\item
  functions
\item
  organization
\item
  package description
\item
  compare similar options
\item
  particular task
\item
  general purpose
\item
  improve base R (efficiency, elegance)
\item
  expand base R (do more)
\item
  What do you want to use the package to accomplish?
\item
  What functionalities of base R could be improved?
\item
  What functionalities would you like to add to base R to expand what
  you can do?
\item
  How do you want to work with inputs and what are the desired outputs?
\end{itemize}

\begin{enumerate}
\def\labelenumi{\arabic{enumi}.}
\setcounter{enumi}{1}
\tightlist
\item
  \textbf{Spend time searching; find and collect options}
\end{enumerate}

\begin{itemize}
\tightlist
\item
  internet searches (keyword ``\ldots in R'')
\item
  textbooks (``{[}x{]} with R'' series)

  \begin{itemize}
  \tightlist
  \item
    \href{https://rstudio.com/resources/books/}{RStudio books}
  \end{itemize}
\item
  tutorials
\item
  courses
\item
  social media (\#rstats)
\item
  conferences (e.g., RStudio, useR!)
\item
  consult collaborators
\item
  CRAN task views (based on certain disciplines and methodologies)
\item
  Research articles

  \begin{itemize}
  \tightlist
  \item
    Check which packages have been used in research in your field
    (provide suggestions for good Google Scholar search queries to
    identify papers that have used certain packages or that present a
    package) Alternatively, check the Methods and References sections of
    papers in your field.
  \item
    Related to that, we could talk about how packages can be cited (the
    \texttt{citation} function produces one in the preferred format for
    any package). You can look up most packages in Google Scholar to see
    how many times it's been cited by looking at the ``Cited by'' link
    with the reference. See for example the first listing at
    https://scholar.google.com/scholar?hl=en\&as\_sdt=0\%2C6\&q=dplyr\&btnG=
  \end{itemize}
\item
  Blogs

  \begin{itemize}
  \tightlist
  \item
    find posts with overviews of new packages
  \item
    \href{https://rstudio.com/products/rpackages/}{RStudio packages}
    developed, maintained, or contributed to by the RStudio team
    categorized by purpose
  \item
    \href{https://rstudio.com/resources/webinars/}{RStudio webinars}
  \item
    \href{https://ropensci.org/packages/}{ROpenSci packages} site
    includes links to blogs, tutorials, and examples for using specific
    packages
  \item
    \href{https://www.r-bloggers.com/}{R-bloggers}
  \item
    \href{https://blog.revolutionanalytics.com/}{Revolutions} daily blog
  \item
    \href{https://rweekly.org/}{R Weekly} has specific sections for new
    packages and updated packages
  \end{itemize}
\item
  People

  \begin{itemize}
  \tightlist
  \item
    Joe Rickert of RStudio (Ambassador at Large) used to regularly
    highlight interesting new packages (check to see if he still does).

    \begin{itemize}
    \tightlist
    \item
      \href{https://rstudio.com/resources/rstudioconf-2018/what-makes-a-great-r-package-joseph-rickert/}{what
      makes a great R package}
    \item
      active on Twitter (\texttt{@RStudioJoe})
    \end{itemize}
  \item
    Mara Averick of RStudio (\texttt{tidyverse} developer advocate)
    advertises cool new R things; check if any focus on packages.

    \begin{itemize}
    \tightlist
    \item
      I cannot find any recent activity or posts by Mara beyond Twitter
      (\texttt{@dataandme}), but some of her sites have interactive info
      about cool packages and how to use their features.
    \item
      There are also several videos and posts from her that relate to
      conferences and unconfs.
    \end{itemize}
  \end{itemize}
\end{itemize}

\begin{enumerate}
\def\labelenumi{\arabic{enumi}.}
\setcounter{enumi}{2}
\tightlist
\item
  \textbf{Check how it's shared}
\end{enumerate}

\begin{itemize}
\tightlist
\item
  check repository association

  \begin{itemize}
  \tightlist
  \item
    CRAN
  \item
    Bioconductor
  \item
    GitHub
  \item
    GitLab (alternative to GitHub)
  \item
    ROpenSci (runs its own repository, only includes ones it has
    peer-reviewed)
  \item
    Self-hosted repositories (can be made with the \texttt{drat}
    package; see paper)
  \item
    purpose of repositories: mechanisms of quality control that
    regularly check code and manage webs of dependencies
  \end{itemize}
\item
  alternative ways R packages can be shared (not repo)

  \begin{itemize}
  \tightlist
  \item
    zipped file
  \item
    collaborators
  \end{itemize}
\end{itemize}

\begin{enumerate}
\def\labelenumi{\arabic{enumi}.}
\setcounter{enumi}{3}
\tightlist
\item
  \textbf{Explore the availability and quality of help}
\end{enumerate}

\begin{itemize}
\tightlist
\item
  help files
\item
  \texttt{help()}
\item
  vignettes
\item
  \texttt{DOCUMENTATION} file
\item
  ``cheatsheets'' from RSudio
\item
  RDocumentation (key word search, task views)
\item
  websites (e.g., \texttt{packagedown})
\item
  \texttt{bookdown} books
\item
  compare documentation completeness and resource quality
\item
  find ways to get help beyond initial documentation
\item
  listservs
\item
  online communities
\item
  Stack Overflow (frequency of questions and answers on the topic)
\item
  See if GitHub repo for the package seems responsive to Issues
\item
  \texttt{Rcpp} is an example of high-quality help

  \begin{itemize}
  \tightlist
  \item
    associated book
  \item
    maintainer, Dirk, is known to be responsive to user questions
    (listserv)
  \item
    ample documentation including examples to get started
  \end{itemize}
\end{itemize}

\begin{enumerate}
\def\labelenumi{\arabic{enumi}.}
\setcounter{enumi}{4}
\tightlist
\item
  \textbf{Verify the credibility of the author(s)}
\end{enumerate}

\begin{itemize}
\tightlist
\item
  team or single author (robust team?)
\item
  associations (e.g., academia, industry, labs)
\item
  expertise
\item
  reputation
\item
  experience (e.g., portfolio of packages, history of R development)
\item
  role in R development (e.g., RStudio, regarded bio labs)
\item
  profiles (e.g., GitHub, Google Scholar, Research Gate, Twitter)
\end{itemize}

\begin{enumerate}
\def\labelenumi{\arabic{enumi}.}
\setcounter{enumi}{5}
\tightlist
\item
  \textbf{Investigate the package development}
\end{enumerate}

\begin{itemize}
\tightlist
\item
  best practices
\item
  unit testing (manage quality control)
\item
  dependencies
\item
  coverage by tests
\item
  number of versions
\item
  clarity of NEWS (explain updates and changes)
\item
  GitHub Issues (history, resolution)
\end{itemize}

\begin{enumerate}
\def\labelenumi{\arabic{enumi}.}
\setcounter{enumi}{6}
\tightlist
\item
  \textbf{Read, research literature, seek evidence of peer review}
\end{enumerate}

\begin{itemize}
\tightlist
\item
  publications
\item
  package itself
\item
  papers about the package
\item
  ROpenSci
\item
  associations with books or publications from scientific publishers
\end{itemize}

\begin{enumerate}
\def\labelenumi{\arabic{enumi}.}
\setcounter{enumi}{7}
\tightlist
\item
  \textbf{Quantify how established the package is}
\end{enumerate}

\begin{itemize}
\tightlist
\item
  dependencies
\item
  versions
\item
  updates
\item
  number of downloads
\item
  popularity
\item
  leaderboard
\item
  ranking systems
\item
  number of citations in Google Scholar
\end{itemize}

\begin{enumerate}
\def\labelenumi{\arabic{enumi}.}
\setcounter{enumi}{8}
\tightlist
\item
  \textbf{Put the package to the test}
\end{enumerate}

\begin{itemize}
\tightlist
\item
  explore code
\item
  interact with trial and error
\item
  get a feel for using it in context of your goal
\item
  open-source framework
\item
  GitHub mirror of CRAN as an alternative to downloading zipped package
  file
\item
  How interoperable it is with other packages that you want to use?
\item
  some packages do what they do really well, but it is hard to use them
  with the tidyverse or other outside packages

  \begin{itemize}
  \tightlist
  \item
    S3 or S4 objects that make it hard to work them into a pipeline
    where their functions are not the last step
  \end{itemize}
\item
  packages that help with interoperability

  \begin{itemize}
  \tightlist
  \item
    \texttt{broom} and \texttt{biobroom}: make it easier to put numerous
    statistical functions into a larger tidyverse workflow
  \item
    Max Kuhn's \texttt{caret} package for machine learning---adds a
    layer that lets you use the same interface to work with machine
    learning algorithms from lots of different packages that otherwise
    all have slightly different interfaces for calling the algorithm and
    working with the results.
  \end{itemize}
\end{itemize}

\begin{enumerate}
\def\labelenumi{\arabic{enumi}.}
\setcounter{enumi}{9}
\tightlist
\item
  \textbf{Develop your own package}
\end{enumerate}

\begin{itemize}
\tightlist
\item
  necessity
\item
  innovative idea
\item
  standardize
\item
  automate
\item
  novel approach or method
\item
  unique and specialized purpose
\item
  copying and pasting functions repeatedly
\item
  collection of functions you use often
\item
  desire to easily share code, data, documentation, and tests with
  others
\item
  \href{https://hilaryparker.com/2013/04/03/personal-r-packages/}{``personal
  R packages''}
\item
  suggested package development resources for readers:

  \begin{itemize}
  \tightlist
  \item
    \href{http://r-pkgs.had.co.nz/}{R Packages: Organize, Test,
    Document, and Share Your Code by Hadley Wickham}
  \item
    \href{https://www.ncbi.nlm.nih.gov/pmc/articles/PMC5390961/pdf/pcbi.1005412.pdf}{Ten
    simple rules for making research software more robust (Taschuk \&
    Wilson)}
  \end{itemize}
\end{itemize}

\hypertarget{references}{%
\section*{References}\label{references}}
\addcontentsline{toc}{section}{References}

\hypertarget{refs}{}
\leavevmode\hypertarget{ref-perez2016ten}{}%
1. Perez-Riverol Y, Gatto L, Wang R, Sachsenberg T, Uszkoreit J, Veiga
Leprevost F da, et al. Ten simple rules for taking advantage of git and
github. PLoS computational biology. Public Library of Science; 2016;12.

\leavevmode\hypertarget{ref-phonenumber}{}%
2. Myles S. Phonenumber: Convert letters to numbers and back as on a
telephone keypad {[}Internet{]}. 2015. Available:
\url{https://CRAN.R-project.org/package=phonenumber}

\leavevmode\hypertarget{ref-dplyr}{}%
3. Wickham H, François R, Henry L, Müller K. Dplyr: A grammar of data
manipulation {[}Internet{]}. 2020. Available:
\url{https://CRAN.R-project.org/package=dplyr}

\leavevmode\hypertarget{ref-datatable}{}%
4. Dowle M, Srinivasan A. Data.table: Extension of `data.frame`
{[}Internet{]}. 2019. Available:
\url{https://CRAN.R-project.org/package=data.table}

\leavevmode\hypertarget{ref-gentleman2004bioconductor}{}%
5. Gentleman RC, Carey VJ, Bates DM, Bolstad B, Dettling M, Dudoit S, et
al. Bioconductor: Open software development for computational biology
and bioinformatics. Genome biology. Springer; 2004;5: R80.

\leavevmode\hypertarget{ref-holmes2018modern}{}%
6. Holmes S, Huber W. Modern statistics for modern biology
{[}Internet{]}. Cambridge University Press; 2018. Available:
\url{https://web.stanford.edu/class/bios221/book/index.html}

\nolinenumbers


\end{document}

