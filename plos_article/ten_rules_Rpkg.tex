% Template for PLoS
% Version 3.5 March 2018
%
% % % % % % % % % % % % % % % % % % % % % %
%
% -- IMPORTANT NOTE
%
% This template contains comments intended
% to minimize problems and delays during our production
% process. Please follow the template instructions
% whenever possible.
%
% % % % % % % % % % % % % % % % % % % % % % %
%
% Once your paper is accepted for publication,
% PLEASE REMOVE ALL TRACKED CHANGES in this file
% and leave only the final text of your manuscript.
% PLOS recommends the use of latexdiff to track changes during review, as this will help to maintain a clean tex file.
% Visit https://www.ctan.org/pkg/latexdiff?lang=en for info or contact us at latex@plos.org.
%
%
% There are no restrictions on package use within the LaTeX files except that
% no packages listed in the template may be deleted.
%
% Please do not include colors or graphics in the text.
%
% The manuscript LaTeX source should be contained within a single file (do not use \input, \externaldocument, or similar commands).
%
% % % % % % % % % % % % % % % % % % % % % % %
%
% -- FIGURES AND TABLES
%
% Please include tables/figure captions directly after the paragraph where they are first cited in the text.
%
% DO NOT INCLUDE GRAPHICS IN YOUR MANUSCRIPT
% - Figures should be uploaded separately from your manuscript file.
% - Figures generated using LaTeX should be extracted and removed from the PDF before submission.
% - Figures containing multiple panels/subfigures must be combined into one image file before submission.
% For figure citations, please use "Fig" instead of "Figure".
% See http://journals.plos.org/plosone/s/figures for PLOS figure guidelines.
%
% Tables should be cell-based and may not contain:
% - spacing/line breaks within cells to alter layout or alignment
% - do not nest tabular environments (no tabular environments within tabular environments)
% - no graphics or colored text (cell background color/shading OK)
% See http://journals.plos.org/plosone/s/tables for table guidelines.
%
% For tables that exceed the width of the text column, use the adjustwidth environment as illustrated in the example table in text below.
%
% % % % % % % % % % % % % % % % % % % % % % % %
%
% -- EQUATIONS, MATH SYMBOLS, SUBSCRIPTS, AND SUPERSCRIPTS
%
% IMPORTANT
% Below are a few tips to help format your equations and other special characters according to our specifications. For more tips to help reduce the possibility of formatting errors during conversion, please see our LaTeX guidelines at http://journals.plos.org/plosone/s/latex
%
% For inline equations, please be sure to include all portions of an equation in the math environment.
%
% Do not include text that is not math in the math environment.
%
% Please add line breaks to long display equations when possible in order to fit size of the column.
%
% For inline equations, please do not include punctuation (commas, etc) within the math environment unless this is part of the equation.
%
% When adding superscript or subscripts outside of brackets/braces, please group using {}.
%
% Do not use \cal for caligraphic font.  Instead, use \mathcal{}
%
% % % % % % % % % % % % % % % % % % % % % % % %
%
% Please contact latex@plos.org with any questions.
%
% % % % % % % % % % % % % % % % % % % % % % % %

\documentclass[10pt,letterpaper]{article}
\usepackage[top=0.85in,left=2.75in,footskip=0.75in]{geometry}

% amsmath and amssymb packages, useful for mathematical formulas and symbols
\usepackage{amsmath,amssymb}

% Use adjustwidth environment to exceed column width (see example table in text)
\usepackage{changepage}

% Use Unicode characters when possible
\usepackage[utf8x]{inputenc}

% textcomp package and marvosym package for additional characters
\usepackage{textcomp,marvosym}

% cite package, to clean up citations in the main text. Do not remove.
% \usepackage{cite}

% Use nameref to cite supporting information files (see Supporting Information section for more info)
\usepackage{nameref,hyperref}

% line numbers
\usepackage[right]{lineno}

% ligatures disabled
\usepackage{microtype}
\DisableLigatures[f]{encoding = *, family = * }

% color can be used to apply background shading to table cells only
\usepackage[table]{xcolor}

% array package and thick rules for tables
\usepackage{array}

% create "+" rule type for thick vertical lines
\newcolumntype{+}{!{\vrule width 2pt}}

% create \thickcline for thick horizontal lines of variable length
\newlength\savedwidth
\newcommand\thickcline[1]{%
  \noalign{\global\savedwidth\arrayrulewidth\global\arrayrulewidth 2pt}%
  \cline{#1}%
  \noalign{\vskip\arrayrulewidth}%
  \noalign{\global\arrayrulewidth\savedwidth}%
}

% \thickhline command for thick horizontal lines that span the table
\newcommand\thickhline{\noalign{\global\savedwidth\arrayrulewidth\global\arrayrulewidth 2pt}%
\hline
\noalign{\global\arrayrulewidth\savedwidth}}


% Remove comment for double spacing
%\usepackage{setspace}
%\doublespacing

% Text layout
\raggedright
\setlength{\parindent}{0.5cm}
\textwidth 5.25in
\textheight 8.75in

% Bold the 'Figure #' in the caption and separate it from the title/caption with a period
% Captions will be left justified
\usepackage[aboveskip=1pt,labelfont=bf,labelsep=period,justification=raggedright,singlelinecheck=off]{caption}
\renewcommand{\figurename}{Fig}

% Use the PLoS provided BiBTeX style
% \bibliographystyle{plos2015}

% Remove brackets from numbering in List of References
\makeatletter
\renewcommand{\@biblabel}[1]{\quad#1.}
\makeatother



% Header and Footer with logo
\usepackage{lastpage,fancyhdr,graphicx}
\usepackage{epstopdf}
%\pagestyle{myheadings}
\pagestyle{fancy}
\fancyhf{}
%\setlength{\headheight}{27.023pt}
%\lhead{\includegraphics[width=2.0in]{PLOS-submission.eps}}
\rfoot{\thepage/\pageref{LastPage}}
\renewcommand{\headrulewidth}{0pt}
\renewcommand{\footrule}{\hrule height 2pt \vspace{2mm}}
\fancyheadoffset[L]{2.25in}
\fancyfootoffset[L]{2.25in}
\lfoot{\today}

%% Include all macros below

\newcommand{\lorem}{{\bf LOREM}}
\newcommand{\ipsum}{{\bf IPSUM}}






\usepackage{forarray}
\usepackage{xstring}
\newcommand{\getIndex}[2]{
  \ForEach{,}{\IfEq{#1}{\thislevelitem}{\number\thislevelcount\ExitForEach}{}}{#2}
}

\setcounter{secnumdepth}{0}

\newcommand{\getAff}[1]{
  \getIndex{#1}{Stats,Math,ERHS}
}

\providecommand{\tightlist}{%
  \setlength{\itemsep}{0pt}\setlength{\parskip}{0pt}}

\begin{document}
\vspace*{0.2in}

% Title must be 250 characters or less.
\begin{flushleft}
{\Large
\textbf\newline{Ten simple rules for selecting an R package} % Please use "sentence case" for title and headings (capitalize only the first word in a title (or heading), the first word in a subtitle (or subheading), and any proper nouns).
}
\newline
% Insert author names, affiliations and corresponding author email (do not include titles, positions, or degrees).
\\
Caroline J. Wendt\textsuperscript{\getAff{Stats}, \getAff{Math}},
G. Brooke Anderson\textsuperscript{\getAff{ERHS}}\textsuperscript{*}\\
\bigskip
\textbf{\getAff{Stats}}Department of Statistics, Colorado State University, Fort Collins,
Colorado, United States of America\\
\textbf{\getAff{Math}}Department of Mathematics, Colorado State University, Fort Collins,
Colorado, United States of America\\
\textbf{\getAff{ERHS}}Department of Environmental \& Radiological Health Sciences, Colorado
State University, Fort Collins, Colorado, United States of America\\
\bigskip
* Corresponding author: Brooke.Anderson@colostate.edu\\
\end{flushleft}
% Please keep the abstract below 300 words
\section*{Abstract}
Write the abstract here.

% Please keep the Author Summary between 150 and 200 words
% Use first person. PLOS ONE authors please skip this step.
% Author Summary not valid for PLOS ONE submissions.
\section*{Author summary}
Write the author summary here.

\linenumbers

% Use "Eq" instead of "Equation" for equation citations.
\emph{Text based on plos sample manuscript, see
\url{http://journals.plos.org/ploscompbiol/s/latex}}

\hypertarget{introduction}{%
\section{Introduction}\label{introduction}}

\emph{Explain what R is and how its package ecosystem works.}

Points:

\begin{itemize}
\tightlist
\item
  Open source project, where many people contribute with their own
  extensions
\item
  Large variation in the quality of different extensions (packages)
\item
  That some R users, particularly new ones, struggle with finding and
  picking which packages to use.
\end{itemize}

\emph{Ideas of 10 things}

Finding packages:

\begin{itemize}
\tightlist
\item
  CRAN task views
\item
  Textbooks (``{[}x{]} with R''). May not be latest\ldots{}
\item
  Google searches, social media (\#rstats)
\item
  Conferences (and online streams of those). RStudio, UseR.
\end{itemize}

Picking a good package:

\begin{itemize}
\tightlist
\item
  On a public repository like CRAN or Bioconductor. Explain more about
  these repositories and what their standards are. Explain their role in
  the community. Give the alternative ways that R packages can be shared
  (GitHub, zipped file posted somewhere else). How these regularly check
  code and help with managing the web of dependencies.
\item
  Quality of the documentation. Types of documentation (help files,
  vignette, packagedown website, bookdown book).
\item
  Coverage by tests. Explain about unit testing and how it can help
  control quality.
\item
  Peer review. ROpenSci. Associated with a peer reviewed paper.
  Associated with a book put out by a scientific publisher?
\item
  Looking up package authors. Is there role in R development (RStudio,
  some big bio labs)? Is the work part of their work from an academic
  lab? Do they have a history of a lot of R development? GitHub profile.
  Google scholar profile. Also, is it a team of developers? Robust team?
\item
  Evidence of established package. Lots of version. Clear NEWS providing
  explanations of changes. History of Issues and those being resolved.
\item
  Exploring the code yourself. How open source framework provides this.
  GitHub mirror of CRAN if you don't want to download the zipped package
  file yourself.
\end{itemize}

Here are two sample references: {[}1,2{]}.

\hypertarget{introduction-1}{%
\section{Introduction}\label{introduction-1}}

R is a language and environment for statistical computing and graphics
that was developed by statisticians and is collaboratively maintained by
an international core group of contributors. Unlike many popular
proprietary languages (e.g., MATLAB, SAS, SPSS), R is highly extensible,
free and open-source software; the user can access and thus change,
extend, and share code for desired applications. Accordingly, a vibrant
community of R users has emerged, many of which engage in the
development of extensions to the functionality of base R software known
as packages. A prominent contributor in the R community, Hadley Wickham,
views functional programming as analogous to following a recipe; to
conceptualize packages, imagine R is the kitchen and packages are the
special gadgets which allow you to cook and bake new recipes. R packages
are coding delectables that enable the user to perform practical tasks
(e.g., wrangling and cleaning data frames, designing interactive apps
for visualizing data, performing dimensionality reduction) and solve
problems (e.g., training regression and classification models, assessing
the beta diversity of a population, analyzing gene expression microarray
data) with interesting techniques.

As a natural consequence of the open-source nature of R, there is
variation in the quality of different packages among the numerous
choices that exist. The advanced R user---having developed an intuition
for their workflow---may tend to be relatively confident when searching
for and selecting packages. By contrast, a common experience that
characterizes learning R at the outset is the struggle to 1) find a
package to accomplish a particular task and 2) choose the best package
to perform that task. Even so, there remain obscure and complicated
problems that morph selecting an R package into a barrier despite
experience.

In coding as in life, we endeavor to make choices that optimize
outcomes. Just as one may go about shopping for shoes, deciding which
graduate program to pursue, or conducting a literature review, there is
a science behind selection. We inform our decisions by assessing,
comparing, and filtering options based on indicators of quality such as
utility, association, and reputation. Likewise, choosing an R package
requires attending to similar details. We outline ten simple rules for
finding and selecting R packages so that you will spend less time
searching for the right tools and more time coding delicious recipes.

\hypertarget{list-of-10-rules}{%
\section{List of 10 rules}\label{list-of-10-rules}}

\hypertarget{currently-in-no-particular-order-and-not-precisely-worded}{%
\subsection{(currently in no particular order and not precisely
worded)}\label{currently-in-no-particular-order-and-not-precisely-worded}}

\begin{enumerate}
\def\labelenumi{\arabic{enumi}.}
\tightlist
\item
  \textbf{Consider your purpose}
\end{enumerate}

\begin{itemize}
\tightlist
\item
  What do you want to use the package to accomplish?
\item
  features
\item
  functions
\item
  organization
\item
  package description
\item
  compare similar options
\end{itemize}

\begin{enumerate}
\def\labelenumi{\arabic{enumi}.}
\setcounter{enumi}{1}
\tightlist
\item
  \textbf{Spend time searching; find and collect options}
\end{enumerate}

\begin{itemize}
\tightlist
\item
  internet searches (keyword ``\ldots in R'')
\item
  textbooks (``{[}x{]} with R'' series)
\item
  tutorials
\item
  courses
\item
  social media (\#rstats)
\item
  conferences (e.g., RStudio, useR!)
\item
  consult collaborators
\item
  CRAN task views
\end{itemize}

\begin{enumerate}
\def\labelenumi{\arabic{enumi}.}
\setcounter{enumi}{2}
\tightlist
\item
  \textbf{Check the repository association}
\end{enumerate}

\begin{itemize}
\tightlist
\item
  purpose: mechanisms of quality control that regularly check code and
  manage webs of dependencies
\item
  CRAN
\item
  Bioconductor
\item
  GitHub
\item
  collaborators
\item
  zipped file
\item
  alternative ways R packages can be shared
\end{itemize}

\begin{enumerate}
\def\labelenumi{\arabic{enumi}.}
\setcounter{enumi}{3}
\tightlist
\item
  \textbf{Explore the availability and quality of help}
\end{enumerate}

\begin{itemize}
\tightlist
\item
  help files
\item
  \texttt{help()}
\item
  vignettes
\item
  \texttt{DOCUMENTATION} file
\item
  ``cheatsheets'' from RSudio
\item
  RDocumentation (key word search, task views)
\item
  websites (e.g., \texttt{packagedown})
\item
  \texttt{bookdown} books
\item
  compare documentation completeness and resource quality
\item
  find ways to get help beyond initial documentation
\item
  listservs
\item
  online communities
\item
  Stack Overflow (frequency of questions and answers on the topic)
\item
  See if GitHub repo for the package seems responsive to Issues
\item
  \texttt{Rcpp} is an example of high-quality help

  \begin{itemize}
  \tightlist
  \item
    associated book
  \item
    maintainer, Dirk, is known to be responsive to user questions
    (listserv)
  \item
    ample documentation including examples to get started
  \end{itemize}
\end{itemize}

\begin{enumerate}
\def\labelenumi{\arabic{enumi}.}
\setcounter{enumi}{4}
\tightlist
\item
  \textbf{Verify the credibility of the author(s)}
\end{enumerate}

\begin{itemize}
\tightlist
\item
  team or single author (robust team?)
\item
  associations (e.g., academia, industry, labs)
\item
  expertise
\item
  reputation
\item
  experience (e.g., portfolio of packages, history of R development)
\item
  role in R development (e.g., RStudio, regarded bio labs)
\item
  profiles (e.g., GitHub, Google Scholar, Research Gate, Twitter)
\end{itemize}

\begin{enumerate}
\def\labelenumi{\arabic{enumi}.}
\setcounter{enumi}{5}
\tightlist
\item
  \textbf{Investigate the package development}
\end{enumerate}

\begin{itemize}
\tightlist
\item
  best practices
\item
  unit testing (manage quality control)
\item
  dependencies
\item
  coverage by tests
\item
  number of versions
\item
  clarity of NEWS (explain updates and changes)
\item
  GitHub Issues (history, resolution)
\end{itemize}

\begin{enumerate}
\def\labelenumi{\arabic{enumi}.}
\setcounter{enumi}{6}
\tightlist
\item
  \textbf{Read, research literature, seek evidence of peer review}
\end{enumerate}

\begin{itemize}
\tightlist
\item
  publications
\item
  package itself
\item
  papers about the package
\item
  ROpenSci
\item
  associations with books or publications from scientific publishers
\end{itemize}

\begin{enumerate}
\def\labelenumi{\arabic{enumi}.}
\setcounter{enumi}{7}
\tightlist
\item
  \textbf{Quantify how established the package is}
\end{enumerate}

\begin{itemize}
\tightlist
\item
  dependencies
\item
  versions
\item
  updates
\item
  number of downloads
\item
  popularity
\item
  leaderboard
\item
  ranking systems
\end{itemize}

\begin{enumerate}
\def\labelenumi{\arabic{enumi}.}
\setcounter{enumi}{8}
\tightlist
\item
  \textbf{Put the package to the test}
\end{enumerate}

\begin{itemize}
\tightlist
\item
  explore code
\item
  interact with trial and error
\item
  get a feel for using it in context of your goal
\item
  open-source framework
\item
  GitHub mirror of CRAN as an alternative to downloading zipped package
  file
\end{itemize}

\begin{enumerate}
\def\labelenumi{\arabic{enumi}.}
\setcounter{enumi}{9}
\tightlist
\item
  \textbf{Develop your own package}
\end{enumerate}

\begin{itemize}
\tightlist
\item
  necessity
\item
  innovative idea
\item
  novel approach or method
\item
  unique and specialized purpose
\end{itemize}

\hypertarget{references}{%
\section*{References}\label{references}}
\addcontentsline{toc}{section}{References}

\hypertarget{refs}{}
\leavevmode\hypertarget{ref-Feynman1963118}{}%
1. Feynman RP, Vernon Jr. FL. The theory of a general quantum system
interacting with a linear dissipative system. Annals of Physics.
1963;24: 118--173.
doi:\href{https://doi.org/10.1016/0003-4916(63)90068-X}{10.1016/0003-4916(63)90068-X}

\leavevmode\hypertarget{ref-Dirac1953888}{}%
2. Dirac PAM. The lorentz transformation and absolute time. Physica.
1953;19: 888--896.
doi:\href{https://doi.org/10.1016/S0031-8914(53)80099-6}{10.1016/S0031-8914(53)80099-6}

\nolinenumbers


\end{document}

