% Template for PLoS
% Version 3.5 March 2018
%
% % % % % % % % % % % % % % % % % % % % % %
%
% -- IMPORTANT NOTE
%
% This template contains comments intended
% to minimize problems and delays during our production
% process. Please follow the template instructions
% whenever possible.
%
% % % % % % % % % % % % % % % % % % % % % % %
%
% Once your paper is accepted for publication,
% PLEASE REMOVE ALL TRACKED CHANGES in this file
% and leave only the final text of your manuscript.
% PLOS recommends the use of latexdiff to track changes during review, as this will help to maintain a clean tex file.
% Visit https://www.ctan.org/pkg/latexdiff?lang=en for info or contact us at latex@plos.org.
%
%
% There are no restrictions on package use within the LaTeX files except that
% no packages listed in the template may be deleted.
%
% Please do not include colors or graphics in the text.
%
% The manuscript LaTeX source should be contained within a single file (do not use \input, \externaldocument, or similar commands).
%
% % % % % % % % % % % % % % % % % % % % % % %
%
% -- FIGURES AND TABLES
%
% Please include tables/figure captions directly after the paragraph where they are first cited in the text.
%
% DO NOT INCLUDE GRAPHICS IN YOUR MANUSCRIPT
% - Figures should be uploaded separately from your manuscript file.
% - Figures generated using LaTeX should be extracted and removed from the PDF before submission.
% - Figures containing multiple panels/subfigures must be combined into one image file before submission.
% For figure citations, please use "Fig" instead of "Figure".
% See http://journals.plos.org/plosone/s/figures for PLOS figure guidelines.
%
% Tables should be cell-based and may not contain:
% - spacing/line breaks within cells to alter layout or alignment
% - do not nest tabular environments (no tabular environments within tabular environments)
% - no graphics or colored text (cell background color/shading OK)
% See http://journals.plos.org/plosone/s/tables for table guidelines.
%
% For tables that exceed the width of the text column, use the adjustwidth environment as illustrated in the example table in text below.
%
% % % % % % % % % % % % % % % % % % % % % % % %
%
% -- EQUATIONS, MATH SYMBOLS, SUBSCRIPTS, AND SUPERSCRIPTS
%
% IMPORTANT
% Below are a few tips to help format your equations and other special characters according to our specifications. For more tips to help reduce the possibility of formatting errors during conversion, please see our LaTeX guidelines at http://journals.plos.org/plosone/s/latex
%
% For inline equations, please be sure to include all portions of an equation in the math environment.
%
% Do not include text that is not math in the math environment.
%
% Please add line breaks to long display equations when possible in order to fit size of the column.
%
% For inline equations, please do not include punctuation (commas, etc) within the math environment unless this is part of the equation.
%
% When adding superscript or subscripts outside of brackets/braces, please group using {}.
%
% Do not use \cal for caligraphic font.  Instead, use \mathcal{}
%
% % % % % % % % % % % % % % % % % % % % % % % %
%
% Please contact latex@plos.org with any questions.
%
% % % % % % % % % % % % % % % % % % % % % % % %

\documentclass[10pt,letterpaper]{article}
\usepackage[top=0.85in,left=2.75in,footskip=0.75in]{geometry}

% amsmath and amssymb packages, useful for mathematical formulas and symbols
\usepackage{amsmath,amssymb}

% Use adjustwidth environment to exceed column width (see example table in text)
\usepackage{changepage}

% Use Unicode characters when possible
\usepackage[utf8x]{inputenc}

% textcomp package and marvosym package for additional characters
\usepackage{textcomp,marvosym}

% cite package, to clean up citations in the main text. Do not remove.
% \usepackage{cite}

% Use nameref to cite supporting information files (see Supporting Information section for more info)
\usepackage{nameref,hyperref}

% line numbers
\usepackage[right]{lineno}

% ligatures disabled
\usepackage{microtype}
\DisableLigatures[f]{encoding = *, family = * }

% color can be used to apply background shading to table cells only
\usepackage[table]{xcolor}

% array package and thick rules for tables
\usepackage{array}

% create "+" rule type for thick vertical lines
\newcolumntype{+}{!{\vrule width 2pt}}

% create \thickcline for thick horizontal lines of variable length
\newlength\savedwidth
\newcommand\thickcline[1]{%
  \noalign{\global\savedwidth\arrayrulewidth\global\arrayrulewidth 2pt}%
  \cline{#1}%
  \noalign{\vskip\arrayrulewidth}%
  \noalign{\global\arrayrulewidth\savedwidth}%
}

% \thickhline command for thick horizontal lines that span the table
\newcommand\thickhline{\noalign{\global\savedwidth\arrayrulewidth\global\arrayrulewidth 2pt}%
\hline
\noalign{\global\arrayrulewidth\savedwidth}}


% Remove comment for double spacing
%\usepackage{setspace}
%\doublespacing

% Text layout
\raggedright
\setlength{\parindent}{0.5cm}
\textwidth 5.25in
\textheight 8.75in

% Bold the 'Figure #' in the caption and separate it from the title/caption with a period
% Captions will be left justified
\usepackage[aboveskip=1pt,labelfont=bf,labelsep=period,justification=raggedright,singlelinecheck=off]{caption}
\renewcommand{\figurename}{Fig}

% Use the PLoS provided BiBTeX style
% \bibliographystyle{plos2015}

% Remove brackets from numbering in List of References
\makeatletter
\renewcommand{\@biblabel}[1]{\quad#1.}
\makeatother



% Header and Footer with logo
\usepackage{lastpage,fancyhdr,graphicx}
\usepackage{epstopdf}
%\pagestyle{myheadings}
\pagestyle{fancy}
\fancyhf{}
%\setlength{\headheight}{27.023pt}
%\lhead{\includegraphics[width=2.0in]{PLOS-submission.eps}}
\rfoot{\thepage/\pageref{LastPage}}
\renewcommand{\headrulewidth}{0pt}
\renewcommand{\footrule}{\hrule height 2pt \vspace{2mm}}
\fancyheadoffset[L]{2.25in}
\fancyfootoffset[L]{2.25in}
\lfoot{\today}

%% Include all macros below

\newcommand{\lorem}{{\bf LOREM}}
\newcommand{\ipsum}{{\bf IPSUM}}

\usepackage{color}
\usepackage{fancyvrb}
\newcommand{\VerbBar}{|}
\newcommand{\VERB}{\Verb[commandchars=\\\{\}]}
\DefineVerbatimEnvironment{Highlighting}{Verbatim}{commandchars=\\\{\}}
% Add ',fontsize=\small' for more characters per line
\usepackage{framed}
\definecolor{shadecolor}{RGB}{248,248,248}
\newenvironment{Shaded}{\begin{snugshade}}{\end{snugshade}}
\newcommand{\AlertTok}[1]{\textcolor[rgb]{0.94,0.16,0.16}{#1}}
\newcommand{\AnnotationTok}[1]{\textcolor[rgb]{0.56,0.35,0.01}{\textbf{\textit{#1}}}}
\newcommand{\AttributeTok}[1]{\textcolor[rgb]{0.77,0.63,0.00}{#1}}
\newcommand{\BaseNTok}[1]{\textcolor[rgb]{0.00,0.00,0.81}{#1}}
\newcommand{\BuiltInTok}[1]{#1}
\newcommand{\CharTok}[1]{\textcolor[rgb]{0.31,0.60,0.02}{#1}}
\newcommand{\CommentTok}[1]{\textcolor[rgb]{0.56,0.35,0.01}{\textit{#1}}}
\newcommand{\CommentVarTok}[1]{\textcolor[rgb]{0.56,0.35,0.01}{\textbf{\textit{#1}}}}
\newcommand{\ConstantTok}[1]{\textcolor[rgb]{0.00,0.00,0.00}{#1}}
\newcommand{\ControlFlowTok}[1]{\textcolor[rgb]{0.13,0.29,0.53}{\textbf{#1}}}
\newcommand{\DataTypeTok}[1]{\textcolor[rgb]{0.13,0.29,0.53}{#1}}
\newcommand{\DecValTok}[1]{\textcolor[rgb]{0.00,0.00,0.81}{#1}}
\newcommand{\DocumentationTok}[1]{\textcolor[rgb]{0.56,0.35,0.01}{\textbf{\textit{#1}}}}
\newcommand{\ErrorTok}[1]{\textcolor[rgb]{0.64,0.00,0.00}{\textbf{#1}}}
\newcommand{\ExtensionTok}[1]{#1}
\newcommand{\FloatTok}[1]{\textcolor[rgb]{0.00,0.00,0.81}{#1}}
\newcommand{\FunctionTok}[1]{\textcolor[rgb]{0.00,0.00,0.00}{#1}}
\newcommand{\ImportTok}[1]{#1}
\newcommand{\InformationTok}[1]{\textcolor[rgb]{0.56,0.35,0.01}{\textbf{\textit{#1}}}}
\newcommand{\KeywordTok}[1]{\textcolor[rgb]{0.13,0.29,0.53}{\textbf{#1}}}
\newcommand{\NormalTok}[1]{#1}
\newcommand{\OperatorTok}[1]{\textcolor[rgb]{0.81,0.36,0.00}{\textbf{#1}}}
\newcommand{\OtherTok}[1]{\textcolor[rgb]{0.56,0.35,0.01}{#1}}
\newcommand{\PreprocessorTok}[1]{\textcolor[rgb]{0.56,0.35,0.01}{\textit{#1}}}
\newcommand{\RegionMarkerTok}[1]{#1}
\newcommand{\SpecialCharTok}[1]{\textcolor[rgb]{0.00,0.00,0.00}{#1}}
\newcommand{\SpecialStringTok}[1]{\textcolor[rgb]{0.31,0.60,0.02}{#1}}
\newcommand{\StringTok}[1]{\textcolor[rgb]{0.31,0.60,0.02}{#1}}
\newcommand{\VariableTok}[1]{\textcolor[rgb]{0.00,0.00,0.00}{#1}}
\newcommand{\VerbatimStringTok}[1]{\textcolor[rgb]{0.31,0.60,0.02}{#1}}
\newcommand{\WarningTok}[1]{\textcolor[rgb]{0.56,0.35,0.01}{\textbf{\textit{#1}}}}


\usepackage{booktabs}
\usepackage{longtable}
\usepackage{array}
\usepackage{multirow}
\usepackage{wrapfig}
\usepackage{float}
\usepackage{colortbl}
\usepackage{pdflscape}
\usepackage{tabu}
\usepackage{threeparttable}
\usepackage{threeparttablex}
\usepackage[normalem]{ulem}
\usepackage{makecell}
\usepackage{xcolor}
\usepackage{booktabs}
\usepackage{longtable}
\usepackage{array}
\usepackage{multirow}
\usepackage{wrapfig}
\usepackage{float}
\usepackage{colortbl}
\usepackage{pdflscape}
\usepackage{tabu}
\usepackage{threeparttable}
\usepackage{threeparttablex}
\usepackage[normalem]{ulem}
\usepackage{makecell}
\usepackage{xcolor}



\usepackage{forarray}
\usepackage{xstring}
\newcommand{\getIndex}[2]{
  \ForEach{,}{\IfEq{#1}{\thislevelitem}{\number\thislevelcount\ExitForEach}{}}{#2}
}

\setcounter{secnumdepth}{0}

\newcommand{\getAff}[1]{
  \getIndex{#1}{Stats,Math,ERHS}
}

\providecommand{\tightlist}{%
  \setlength{\itemsep}{0pt}\setlength{\parskip}{0pt}}

\begin{document}
\vspace*{0.2in}

% Title must be 250 characters or less.
\begin{flushleft}
{\Large
\textbf\newline{Ten simple rules for selecting an R package} % Please use "sentence case" for title and headings (capitalize only the first word in a title (or heading), the first word in a subtitle (or subheading), and any proper nouns).
}
\newline
% Insert author names, affiliations and corresponding author email (do not include titles, positions, or degrees).
\\
Caroline J. Wendt\textsuperscript{\getAff{Stats}, \getAff{Math}},
G. Brooke Anderson\textsuperscript{\getAff{ERHS}}\textsuperscript{*}\\
\bigskip
\textbf{\getAff{Stats}}Department of Statistics, Colorado State University, Fort Collins,
Colorado, United States of America\\
\textbf{\getAff{Math}}Department of Mathematics, Colorado State University, Fort Collins,
Colorado, United States of America\\
\textbf{\getAff{ERHS}}Department of Environmental \& Radiological Health Sciences, Colorado
State University, Fort Collins, Colorado, United States of America\\
\bigskip
* Corresponding author: Brooke.Anderson@colostate.edu\\
\end{flushleft}
% Please keep the abstract below 300 words
\section*{Abstract}
R is an increasingly preferred software environment for data analytics
and statistical computing among scientists and practitioners. Packages
markedly extend R's utility and ameliorate inefficient solutions. We
outline ten simple rules for finding relevant packages and determining
which is optimal for your desired use.

% Please keep the Author Summary between 150 and 200 words
% Use first person. PLOS ONE authors please skip this step.
% Author Summary not valid for PLOS ONE submissions.
\section*{Author summary}
Write the author summary here. Do we want to include and author summary?

\linenumbers

% Use "Eq" instead of "Equation" for equation citations.
\emph{Text based on plos sample manuscript, see
\url{http://journals.plos.org/ploscompbiol/s/latex}}

\hypertarget{disclaimer}{%
\section{Disclaimer?}\label{disclaimer}}

Do we need to include a disclaimer in the margin like the one from
{[}1{]} that states: ``\textbf{Competing Interests}: The authors have no
affiliation with GitHub, nor with any other commercial entity mentioned
in this article. The views described here reflect their own views
without input from any third party organization.''

\begin{itemize}
\tightlist
\item
  RStudio
\item
  ROpenSci
\end{itemize}

\hypertarget{funding-acknowledgment}{%
\section{Funding acknowledgment?}\label{funding-acknowledgment}}

Do we need to include a funding acknowledgment in the margin as in the
examples?

\hypertarget{introduction}{%
\section{Introduction}\label{introduction}}

R is a language and environment for statistical computing and graphics
that was developed by statisticians and is collaboratively maintained by
an international core group of contributors {[}2{]}. Unlike several
popular proprietary languages (e.g., MATLAB, SAS, SPSS), R is highly
extensible, free and open-source software; the user can access and thus
change, extend, and share code for desired applications. Accordingly, a
vibrant community of R users has emerged, many of which engage in the
development of extensions to the functionality of base R software known
as packages. There are plenty of analogies in computing that draw
comparisons between programming and culinary arts: recipe structures,
coding cookbooks, and the like. To conceptualize packages, imagine you
are the chef, R is the kitchen, and packages are the special gadgets
which allow you to cook and bake new recipes. R packages are coding
delectables that enable the user to perform practical tasks and solve
problems with interesting techniques.

Are there R packages for wrangling and cleaning data frames, designing
interactive apps for data visualization, or performing dimensionality
reduction? Yes! How do you find an R package that will help you train
regression and classification models, assess the beta diversity of a
population, or analyze gene expression microarray data? The answer is
not as simple; there are tens of thousands of R packages. As a natural
consequence of the open-source nature of R, there is variation in the
quality of different packages among the numerous choices that exist. The
advanced R user---having developed an intuition for their workflow---may
tend to be relatively confident when searching for and selecting
packages. By contrast, a common experience that characterizes learning R
at the outset is the struggle to 1) find a package to accomplish a
particular task or solve a problem of interest and 2) choose the best
package to perform that task. Even so, there remain obscure and
complicated problems that morph selecting an R package into a barrier
despite experience.

In coding as in life, we endeavor to make choices that optimize
outcomes. Just as one may go about shopping for shoes, deciding which
graduate program to pursue, or conducting a literature review, there is
a science behind selection. We inform our decisions by assessing,
comparing, and filtering options based on indicators of quality such as
utility, association, and reputation. Likewise, choosing an R package
requires attending to similar details. We outline ten simple rules for
finding and selecting R packages so that you will spend less time
searching for the right tools and more time coding delicious recipes.

\hypertarget{rule-1-consider-your-purpose}{%
\section{Rule 1: Consider your
purpose}\label{rule-1-consider-your-purpose}}

There are often several different ways to accomplish a task or arrive at
a solution while programming, albeit some ways are more elegant and
efficient than others. To optimize your workflow, consider your purpose
by 1) identifying your task to understand what you are trying to do and
2) defining the scope of the task to determine how you are going to do
it. For some tasks, coding your own recipe with existing tools is
practical, while other tasks benefit from new tools.

If the scope of your task is simple or reasonable given your knowledge
and skills, using an R package may not be appropriate. That is, some
instances do not warrant additional functions, data, and documentation.
There are advantages of coding in base R to implement a unique solution
for your problem. In particular, when you code from scratch, you know
precisely what you are running; thus, your script may be easier to
decipher and maintain. Conversely, packages require reliance on shared
code with features of which you may not be aware.

Packages are valuable when a task has a broad scope or is beyond the
scope of what you desire to code in base R; a task that is narrow in
scope is not necessarily simple. While there are ways to cook up an
algorithm using for loops and conditionals in base R, a relevant package
may accomplish the same goal in a more reproducible manner with less
code and fewer bugs. In general, the more reasonable it is for a given
task to be abstracted away from its context, the more plausible it is
that someone has generalized its themes, developed efficient algorithms,
and organized them in an intuitive way to share with other R users.
Extensive tasks justify sophisticated frameworks with several functions
that form a cohesive package. Numerous processes that involve
data---although varying in application---are ubiquitous. Data
manipulation is one such common task that has been streamlined by
packages such as \texttt{dplyr} and \texttt{tidyr} {[}3,4{]} (See Table
1). Nevertheless, there are indeed packages for seemingly singular
tasks, which you may favor over coding from scratch. Some R packages are
small and have a specific use conducive to tasks that require highly
specialized functions. For example, there is a package for converting
English letters to numbers as on a telephone keypad {[}5{]}.

If you define your purpose by making observations about and considering
limitations of your current toolbox before you start searching for new
tools, you will be more likely to recognize what you do (and do not)
need. Which existing functionalities in base R could be improved in
context of your problem? Which new functionalities would you like to add
to base R to expand what you can do? Develop a list of domain-specific
keywords that relate what you are trying to do to how one may go about
doing it to narrow your search. Identify the type of inputs you have and
envision working with them; contemplate the desired outputs and
corresponding format. For instance, suppose you are using Bioconductor
packages in analyses and have data you would like to visualize; you must
consider that the inputs will be of a certain class---namely, S3 or S4
objects---that impose restrictions when creating graphics. The data
visualization package you use should address such limitations.

\hypertarget{rule-2-spend-time-searching-find-and-collect-options}{%
\section{Rule 2: Spend time searching; find and collect
options}\label{rule-2-spend-time-searching-find-and-collect-options}}

Those who have used R packages may know that although leveraging
existing tools can be advantageous, the initial challenge of finding a
suitable package for a given task can obstruct potential benefits.
Relatedly, new R users who are unfamiliar with the structure and syntax
of the language may be hindered by the process of finding R packages
because they do not know where to search, what to look for, nor how to
sift through options. R packages are mentioned in a variety of places
online, in print, and elsewhere. You can discover new R packages anytime
you learn R-related topics, collaborate with other R users, or browse
the internet.

\textbf{Learn}

Packages are essential to venturing beyond base R and thus quickly
become an integral aspect of advancing your R skills. When learning how
to program in R, you are typically introduced to some of the most common
packages, which tend to have more general purposes (\textbf{Table 1}).
In addition, reputable online tutorials, courses, and books are helpful
resources for acquiring knowledge about packages that are versatile and
reliable---many of which are short, accessible, and either affordable or
offered at no cost to the learner. We recommend online R programming
courses such as those through
\href{https://www.coursera.org/learn/r-programming}{Coursera} and
\href{https://www.codecademy.com/learn/learn-r}{Codeacademy} for
interactive learning and R book series including the
\href{https://rstudio.com/resources/books/}{RStudio books} and Springer
titles for further reading.

\textbf{Collaborate}

An inclusive and collaborative community is an overlooked, yet integral
aspect of a software's success {[}6{]}. A defining feature of R is the
enthusiasm of its users and contributors alike. The R community has a
widespread internet presence across various platforms; however, members
are markedly active on Twitter, a place where R users seek help, share
ideas, and stay informed on \texttt{\#rstats} happenings including
releases of new packages {[}7{]}. Beyond social media, numerous featured
pages and R blogs serve as another informal and up-to-date, yet more
detailed venue for communicating and promoting R-related information
(\textbf{Table ?}). Notably, Joseph Rickert, Ambassador at Large for
RStudio, writes monthly posts on the
\href{https://rviews.rstudio.com/}{R Views} blog highlighting
exceptional new R packages in addition to special articles about
recently released packages and lists of top packages within certain
categories (e.g., Computational Methods, Data, Machine Learning,
Medicine, Science, Statistics, Time Series, Utilities, Visualization).

A developer of an R package may intend for it to be private (exclusively
for personal or professional use) or public (free and available for use
by anyone) {[}8{]}. If your task is specific to a line of research,
consult colleagues to see if they have relevant (private) code they
would be willing to share. Alternatively, literature in your field may
either introduce R packages developed to solve a unique data science
problem or mention packages used during the research process. The former
may be published in the \emph{Journal of Statistical Software},
\emph{The R Journal}, or \emph{BMC Bioinformatics}, for example, and
search queries that include \texttt{"R\ package"} along with domain
keywords will narrow results. The latter requires identifying authors
whom have used R in their analyses, hence packages may be mentioned in
the Methods and/or References sections of the article. Accordingly,
formatted citations for R packages can be obtained in R with
\texttt{citation(package\ \ =\ "\ldots{}")}. You can search for packages
directly by name in Google Scholar: the \texttt{Cited\ by} link displays
the number of times a package has been cited which connects to a page
with those publications. Lastly, conferences are another collaborative
environment wherein you can learn about R packages. There are two major
annual R conferences,
\href{https://rstudio.com/conference/}{rstudio::conf} for industry and
useR! for academia; conferences in your field may foster connections
with fellow scientists whom use R for similar tasks and help you collect
information about packages related to your expertise. Talks and
presentations at conferences are often recorded and made available
online for playback at a later date.

\textbf{Browse}

Based on prior experience---not unlike solution-seeking for many tasks
nowadays---you may think that finding R packages relies heavily on
internet search queries. Indeed, search engines such as Google return
ample pages related to anything \texttt{"\ldots{}in\ R"}. However, this
approach can lead to frustration and confusion when attempting to find a
package for your purpose (see Rule 1). Instead, we recommend initially
searching for packages in repositories such as the Comprehensive R
Archive Network (CRAN), GitHub, or Bioconductor, all of which will be
further discussed in Rule 3. In particular,
\href{https://cran.r-project.org/web/views/}{CRAN Task Views} are
concentrated topics from certain disciplines and methodologies related
to statistical computing that group R packages by the tasks they perform
(e.g., Econometrics, Genetics, Optimization, Spatial). In the HTML
version, you can browse alphabetized subcategories within each Task View
and read concise descriptions to find tools with specific functions.
Alternatively, you can access Task Views directly from the R console
with \texttt{ctv::CRAN.views()}. To date, there are 41 Task Views that
collectively contain thousands of packages which are curated and
regularly tested. Moreover, CRAN Task Views provide tools that enable
the user to automatically install all packages within a targeted area of
interest. Ultimately, CRAN Task Views address several major user-end
issues that have arisen due to the extensive amount of available
packages by providing task-based organization, easy simultaneous
installation of related packages, meta-information, ensured maintenance,
and quality control {[}9{]}. Another place to find well-maintained tools
that promote reusable software and reproducibility when working with
scientific data in research applications is through
\href{https://ropensci.org/packages/}{rOpenSci packages}. Packages are
organized by name, maintainer, description, and status (i.e., activity,
association, review), can be filtered according to purpose, and searched
by name, maintainer, or keywords.

\hypertarget{rule-3-check-how-its-shared}{%
\section{Rule 3: Check how it's
shared}\label{rule-3-check-how-its-shared}}

\hypertarget{rule-4-explore-the-availability-and-quality-of-help}{%
\section{Rule 4: Explore the availability and quality of
help}\label{rule-4-explore-the-availability-and-quality-of-help}}

\hypertarget{rule-5-verify-the-credibility-of-the-authors}{%
\section{Rule 5: Verify the credibility of the
author(s)}\label{rule-5-verify-the-credibility-of-the-authors}}

\hypertarget{rule-6-investigate-the-package-development}{%
\section{Rule 6: Investigate the package
development}\label{rule-6-investigate-the-package-development}}

\hypertarget{rule-7-read-research-literature-seek-evidence-of-peer-review}{%
\section{Rule 7: Read, research literature, seek evidence of peer
review}\label{rule-7-read-research-literature-seek-evidence-of-peer-review}}

\hypertarget{rule-8-quantify-how-established-the-package-is}{%
\section{Rule 8: Quantify how established the package
is}\label{rule-8-quantify-how-established-the-package-is}}

\hypertarget{rule-9-put-the-package-to-the-test}{%
\section{Rule 9: Put the package to the
test}\label{rule-9-put-the-package-to-the-test}}

\hypertarget{rule-10-develop-your-own-package}{%
\section{Rule 10: Develop your own
package}\label{rule-10-develop-your-own-package}}

Alternative solutions can be sought when a package to solve your data
science problem is nonexistent. An R package is the fundamental unit of
shareable code; rather than exclusively being a user of packages, you
can create them---more easily than you may think {[}10{]}. Just as there
are numerous R packages for distinct tasks, the reasons why you might
want to create a package are abundant: necessity, innovation,
standardization, automation, specialty, containment, organization,
sharing, collaboration, extensibility, etc.

Whatever your motivation, R packages are simply toolkits; you can create
a package out of any collection of specialty functions. Packages need
not be formal nor entirely cohesive. For instance, personal R packages
(e.g., \texttt{Hmisc} and \texttt{broman}) are comprised of
miscellaneous functions which the creator has developed and frequently
uses {[}11,12{]}. Functions are necessary for efficiency and warranted
when you repetitiously copy and paste your code while making slight
modifications after each iteration {[}13{]}. The concept of personal R
packages demonstrates a unique purpose for packages beyond the
conventional. R packages are not solely reserved for specific tasks with
comprehensive methods; rather, package development can help you learn
how to apply proper coding techniques to writing functions and
documentation with reproducibility and collaboration in mind {[}14{]}.

Although you may not anticipate that anyone else will use your tools,
following best practices for package development will yield more
favorable outcomes. As a consumer of shared packages, you know the
inherent benefits of robust software development relative to the quality
of code, data, documentation, versions, and tests {[}15{]}. Similarly,
creating a valuable package for personal use requires consideration for
your future self and anticipation of distributing your code should the
need arise. Consider using version control and take advantage of
existing resources. Indeed, there are R packages that aid in package
development (e.g., \texttt{devtools}, \texttt{usethis},
\texttt{testthat}, \texttt{roxygen2}, \texttt{rlang}, \texttt{drat}).
There is no lack of effective organizational frameworks to reference in
the open source R community; in fact, repositories for many exemplary
packages are available on GitHub. We recommend consulting resources
authored by expert R developers including \href{https://r-pkgs.org/}{R
Packages} by Hadley Wickham and Jennifer Bryan as well as the official
manual,
\href{https://cran.r-project.org/doc/manuals/r-release/R-exts.html}{Writing
R Extensions}, from CRAN {[}10,16{]}.

\hypertarget{conclusion}{%
\section{Conclusion}\label{conclusion}}

Computational reproducibility is surfacing as a central axiom in
academia as researchers identify the need for means by which they can
implement transparent systems {[}17,18{]}. It follows that former
approaches and traditional methods tend to be at odds with productivity
and collaboration; much of the variability in science can be attributed
to differences in workflow such that the absence of automation is deemed
irresponsible {[}19{]}. The open source R language has become the
dominant quantitative programming environment in academic statistics,
enabling researchers to share workflows and reexecute scripts within and
across subsets of the scientific community {[}19{]}. R is increasingly
used by researchers in computational biology and bioinformatics, a
discipline among many that is generating extensive heterogenous and
complex data that demands standard tools and rigorous methods that beget
reproducibility {[}20,21{]}. More broadly, as the R ecosystem---in which
the life of modern data analysis thrives---rapidly evolves alongside the
burgeoning R community, R is exhibiting sustained growth when compared
to similar languages, particularly in academia, healthcare, and
government {[}22{]}.

R packages are a defining feature of the language insofar as many are
robust and easily learnable. Some of the most prominent R packages are a
result of the developer abstracting common elements of a data science
problem into a workflow that can be shared and accompanied by thorough
descriptions of the process and purpose. In this way, R packages have
effectively transformed how we interact with data in the modern day in,
perhaps, a more impactful manner than many revered contributions to
theoretical statistics {[}19{]}. Packages greatly enhance the user
experience and enable you to be more efficient and effective at learning
from data regardless of prior experience. However, the sheer quantity
and potential complexity of available R packages can undermine their
collective benefits. Finding and choosing packages, particularly for
beginners, can be daunting and convoluted. R users often struggle to
sift through the tools at their disposal and wonder how to distinguish
appropriate usage. These ten simple rules for navigating the shared code
in the R community are intended to serve as a valuable page in your
computing cookbook---one that will evolve into intuition yet remain a
reliable reference. May searching for and selecting proper tools no
longer spoil your appetite and dissuade you from discovering, trying,
creating, and sharing new recipes.

\hypertarget{table-1-general-packages}{%
\section{Table 1 (general packages)}\label{table-1-general-packages}}

\begin{Shaded}
\begin{Highlighting}[]
\KeywordTok{library}\NormalTok{(kableExtra)}
\KeywordTok{library}\NormalTok{(knitr)}
\end{Highlighting}
\end{Shaded}

\begin{Shaded}
\begin{Highlighting}[]
\CommentTok{# general packages data}
\NormalTok{gen_pkgs <-}\StringTok{ }\KeywordTok{data.frame}\NormalTok{(}
  \DataTypeTok{Package =} \KeywordTok{c}\NormalTok{(}\StringTok{"readr[note]"}\NormalTok{, }
              \StringTok{"dplyr[note]"}\NormalTok{,}
              \StringTok{"tidyr"}\NormalTok{,}
              
              \StringTok{"broom[note]"}\NormalTok{,}
              \StringTok{"purrr[note]"}\NormalTok{,}
              \StringTok{"caret"}\NormalTok{, }
              \StringTok{"keras"}\NormalTok{, }
              
              \StringTok{"ggplot2[note]"}\NormalTok{, }
              \StringTok{"kableExtra"}\NormalTok{, }
              \StringTok{"rmarkdown"}\NormalTok{),}
  
  \DataTypeTok{Description =} \KeywordTok{c}\NormalTok{(}\StringTok{"read rectangular data (e.g., csv, tsv, and fwf)"}\NormalTok{, }
                  \StringTok{"grammar of data manipulation"}\NormalTok{,}
                  \StringTok{"create tidy data"}\NormalTok{,}
                  
                  \StringTok{"tidy model output"}\NormalTok{, }
                  \StringTok{"functional programming tools"}\NormalTok{,}
                  \StringTok{"train classification and regression models"}\NormalTok{, }
                  \StringTok{"R interface to a neural network library"}\NormalTok{, }
                  
                  \StringTok{"data visualization"}\NormalTok{, }
                  \StringTok{"tables"}\NormalTok{,}
                  \StringTok{"reports"}\NormalTok{),}
  
  \DataTypeTok{Year =} \KeywordTok{c}\NormalTok{(}\StringTok{"readr"}\NormalTok{,}
           \StringTok{"dplyr"}\NormalTok{,}
           \StringTok{"tidyr"}\NormalTok{,}
  
           \StringTok{"broom"}\NormalTok{,}
           \StringTok{"purrr"}\NormalTok{,}
           \StringTok{"caret"}\NormalTok{, }
           \StringTok{"keras"}\NormalTok{,}
                    
           \StringTok{"ggplot2"}\NormalTok{,}
           \StringTok{"kableExtra"}\NormalTok{, }
           \StringTok{"rmarkdown"}\NormalTok{),}
  
  \DataTypeTok{Author =} \KeywordTok{c}\NormalTok{(}\StringTok{"readr Wickham et al."}\NormalTok{,}
             \StringTok{"dplyr Wickham et al."}\NormalTok{, }
             \StringTok{"tidyr Wickham et al."}\NormalTok{,}
  
             \StringTok{"broom Robinson et al."}\NormalTok{,}
             \StringTok{"purrr Henry et al."}\NormalTok{,}
             \StringTok{"caret Kuhn et al."}\NormalTok{, }
             \StringTok{"keras Falbel et al."}\NormalTok{,}
                    
             \StringTok{"ggplot2 Wickham et al."}\NormalTok{,}
             \StringTok{"kableExtra Zhu et al."}\NormalTok{, }
             \StringTok{"rmarkdown Allaire et al."}\NormalTok{),}
  
  \DataTypeTok{Documentation =} \KeywordTok{c}\NormalTok{(}\StringTok{"readr"}\NormalTok{, }
                    \StringTok{"dplyr"}\NormalTok{, }
                    \StringTok{"tidyr"}\NormalTok{,}
  
                    \StringTok{"broom"}\NormalTok{,}
                    \StringTok{"purrr"}\NormalTok{,}
                    \StringTok{"https://topepo.github.io/caret/index.html"}\NormalTok{, }
                    \StringTok{"keras"}\NormalTok{,}
                    
                    \StringTok{"ggplot2"}\NormalTok{,}
                    \StringTok{"kableExtra"}\NormalTok{, }
                    \StringTok{"rmarkdown"}\NormalTok{)}
\NormalTok{)}
\end{Highlighting}
\end{Shaded}

\begin{Shaded}
\begin{Highlighting}[]
\CommentTok{# general packages table}
\KeywordTok{kable}\NormalTok{(gen_pkgs, }\DataTypeTok{format =} \StringTok{"latex"}\NormalTok{, }\DataTypeTok{booktabs =} \OtherTok{TRUE}\NormalTok{) }\OperatorTok
\StringTok{  }\CommentTok{# scale}
\StringTok{  }\KeywordTok{kable_styling}\NormalTok{(}\DataTypeTok{latex_options =} \StringTok{"scale_down"}\NormalTok{) }\OperatorTok
\StringTok{  }\CommentTok{# separate rows by category}
\StringTok{  }\KeywordTok{pack_rows}\NormalTok{(}\StringTok{"Data Manipulation"}\NormalTok{, }\DecValTok{1}\NormalTok{, }\DecValTok{3}\NormalTok{) }\OperatorTok\StringTok{ }
\StringTok{  }\KeywordTok{pack_rows}\NormalTok{(}\StringTok{"Statistical Modeling"}\NormalTok{, }\DecValTok{4}\NormalTok{, }\DecValTok{7}\NormalTok{) }\OperatorTok\StringTok{ }
\StringTok{  }\KeywordTok{pack_rows}\NormalTok{(}\StringTok{"Data Visualization"}\NormalTok{, }\DecValTok{8}\NormalTok{, }\DecValTok{10}\NormalTok{) }\OperatorTok
\StringTok{  }\CommentTok{# column wrap}
\StringTok{  }\KeywordTok{column_spec}\NormalTok{(}\DecValTok{1}\NormalTok{, }\DataTypeTok{width =} \StringTok{"10em"}\NormalTok{) }\OperatorTok\StringTok{ }
\StringTok{  }\KeywordTok{column_spec}\NormalTok{(}\DecValTok{2}\NormalTok{, }\DataTypeTok{width =} \StringTok{"20em"}\NormalTok{) }\OperatorTok\StringTok{ }
\StringTok{  }\CommentTok{# bold column names}
\StringTok{  }\KeywordTok{row_spec}\NormalTok{(}\DecValTok{0}\NormalTok{, }\DataTypeTok{bold =}\NormalTok{ T) }\OperatorTok\StringTok{ }
\StringTok{  }\KeywordTok{add_footnote}\NormalTok{(}\KeywordTok{c}\NormalTok{(}\StringTok{"See the tidyverse"}\NormalTok{,}
                 \StringTok{"See the tidyverse"}\NormalTok{,}
                 \StringTok{"See the biobroom analog in Bioconductor"}\NormalTok{,}
                 \StringTok{"See the tidyverse"}\NormalTok{,}
                 \StringTok{"See the tidyverse"}\NormalTok{),}
               \DataTypeTok{notation =} \StringTok{"symbol"}\NormalTok{)}
\end{Highlighting}
\end{Shaded}

\begin{table}[H]
\centering
\resizebox{\linewidth}{!}{
\begin{tabular}{>{\raggedright\arraybackslash}p{10em}>{\raggedright\arraybackslash}p{20em}lll}
\toprule
\textbf{Package} & \textbf{Description} & \textbf{Year} & \textbf{Author} & \textbf{Documentation}\\
\midrule
\addlinespace[0.3em]
\multicolumn{5}{l}{\textbf{Data Manipulation}}\\
\hspace{1em}readr\textsuperscript{*} & read rectangular data (e.g., csv, tsv, and fwf) & readr & readr Wickham et al. & readr\\
\hspace{1em}dplyr\textsuperscript{\dag} & grammar of data manipulation & dplyr & dplyr Wickham et al. & dplyr\\
\hspace{1em}tidyr & create tidy data & tidyr & tidyr Wickham et al. & tidyr\\
\addlinespace[0.3em]
\multicolumn{5}{l}{\textbf{Statistical Modeling}}\\
\hspace{1em}broom\textsuperscript{\ddag} & tidy model output & broom & broom Robinson et al. & broom\\
\hspace{1em}purrr\textsuperscript{\S} & functional programming tools & purrr & purrr Henry et al. & purrr\\
\hspace{1em}caret & train classification and regression models & caret & caret Kuhn et al. & https://topepo.github.io/caret/index.html\\
\hspace{1em}keras & R interface to a neural network library & keras & keras Falbel et al. & keras\\
\addlinespace[0.3em]
\multicolumn{5}{l}{\textbf{Data Visualization}}\\
\hspace{1em}ggplot2\textsuperscript{\P} & data visualization & ggplot2 & ggplot2 Wickham et al. & ggplot2\\
\hspace{1em}kableExtra & tables & kableExtra & kableExtra Zhu et al. & kableExtra\\
\hspace{1em}rmarkdown & reports & rmarkdown & rmarkdown Allaire et al. & rmarkdown\\
\bottomrule
\multicolumn{5}{l}{\textsuperscript{*} See the tidyverse}\\
\multicolumn{5}{l}{\textsuperscript{\dag} See the tidyverse}\\
\multicolumn{5}{l}{\textsuperscript{\ddag} See the biobroom analog in Bioconductor}\\
\multicolumn{5}{l}{\textsuperscript{\S} See the tidyverse}\\
\multicolumn{5}{l}{\textsuperscript{\P} See the tidyverse}\\
\end{tabular}}
\end{table}

\begin{Shaded}
\begin{Highlighting}[]
\CommentTok{## trying to separate color; striped by group}
\CommentTok{#  row_spec(1:3 - 1, extra_latex_after = "\textbackslash{}\textbackslash{}rowcolor\{gray!6\}")}
\CommentTok{#  row_spec(0:3, extra_latex_after = "\textbackslash{}\textbackslash{}rowcolor\{orange!6\}") %>% }
\CommentTok{#  row_spec(4:6, extra_latex_after = "\textbackslash{}\textbackslash{}rowcolor\{gray!6\}") %>% }
\CommentTok{#  row_spec(7:11, extra_latex_after = "\textbackslash{}\textbackslash{}rowcolor\{gray!6\}")}

\CommentTok{## QUESTIONS }
\CommentTok{# Code font for package names in " "? \textbackslash{}texttt\{\}?}
\CommentTok{# How do you repeat same symbol on multiple items with one footnote?}
\CommentTok{# How do you separate colors and stripe by group?}
\CommentTok{# Add title}
\CommentTok{# Add caption}
\CommentTok{# Cite packages in bib and add references in table?}
\CommentTok{# Embed url link to package documentation? Do we want to link cheatsheets?}
\CommentTok{# How do you add link/reference to Table 1 in text in the template?}
\CommentTok{# How do you hide code for table in knitted pdf...include=FALSE errors?}
\CommentTok{# Title for column 2: description/purpose/usage?}
\CommentTok{# Length of description/purpose/usage for each package?}
\end{Highlighting}
\end{Shaded}

\hypertarget{supporting-information}{%
\section{Supporting information}\label{supporting-information}}

Do we need to include any supporting information?

\hypertarget{acknowledgements}{%
\section{Acknowledgements}\label{acknowledgements}}

{[}Acknowledgement of people who have helped{]}

{[}Funding acknowledgement{]}

\hypertarget{references}{%
\section*{References}\label{references}}
\addcontentsline{toc}{section}{References}

\hypertarget{refs}{}
\leavevmode\hypertarget{ref-perez2016}{}%
1. Perez-Riverol Y, Gatto L, Wang R, Sachsenberg T, Uszkoreit J, Veiga
Leprevost F da, et al. Ten simple rules for taking advantage of git and
github. PLoS computational biology. Public Library of Science; 2016;12.

\leavevmode\hypertarget{ref-Rproject2020}{}%
2. Team RC. The r project for statistical computing {[}Internet{]}. The
R Foundation; 2020. Available: \url{https://www.r-project.org/}

\leavevmode\hypertarget{ref-dplyr}{}%
3. Wickham H, François R, Henry L, Müller K. Dplyr: A grammar of data
manipulation {[}Internet{]}. 2020. Available:
\url{https://CRAN.R-project.org/package=dplyr}

\leavevmode\hypertarget{ref-tidyr}{}%
4. Wickham H, Henry L. Tidyr: Tidy messy data {[}Internet{]}. 2020.
Available: \url{https://CRAN.R-project.org/package=tidyr}

\leavevmode\hypertarget{ref-phonenumber}{}%
5. Myles S. Phonenumber: Convert letters to numbers and back as on a
telephone keypad {[}Internet{]}. 2015. Available:
\url{https://CRAN.R-project.org/package=phonenumber}

\leavevmode\hypertarget{ref-smith2017}{}%
6. Smith D. The r community is one of r's best features {[}Internet{]}.
Revolutions. Microsoft; 2017. Available:
\url{https://blog.revolutionanalytics.com/2017/06/r-community.html}

\leavevmode\hypertarget{ref-ellis2017}{}%
7. Ellis SE. Hey! You there! You are welcome here {[}Internet{]}.
rOpenSci. NumFOCUS; 2017. Available:
\url{https://ropensci.org/blog/2017/06/23/community/}

\leavevmode\hypertarget{ref-rickert2018}{}%
8. Rickert J. What makes a great r package? {[}Internet{]}. RStudio;
2018. Available:
\url{https://rstudio.com/resources/rstudioconf-2018/what-makes-a-great-r-package-joseph-rickert/}

\leavevmode\hypertarget{ref-zeileis2005}{}%
9. Zeileis A. CRAN task views. R News. 2005;5: 39--40.

\leavevmode\hypertarget{ref-wickham2015}{}%
10. Wickham H. R packages: Organize, test, document, and share your
code. "O'Reilly Media, Inc."; 2015.

\leavevmode\hypertarget{ref-Hmisc}{}%
11. Harrell Jr FE, Charles Dupont, others. Hmisc: Harrell miscellaneous
{[}Internet{]}. 2020. Available:
\url{https://CRAN.R-project.org/package=Hmisc}

\leavevmode\hypertarget{ref-broman}{}%
12. Broman KW. Broman: Karl broman's r code {[}Internet{]}. 2020.
Available: \url{https://CRAN.R-project.org/package=broman}

\leavevmode\hypertarget{ref-wickham2014}{}%
13. Wickham H. Advanced r. CRC press; 2014.

\leavevmode\hypertarget{ref-parker2013}{}%
14. Parker H. Personal r packages {[}Internet{]}. 2013. Available:
\url{https://hilaryparker.com/2013/04/03/personal-r-packages/}

\leavevmode\hypertarget{ref-taschuk2017}{}%
15. Taschuk M, Wilson G. Ten simple rules for making research software
more robust. PLoS computational biology. Public Library of Science;
2017;13.

\leavevmode\hypertarget{ref-Rcore2020}{}%
16. Team RC. Writing r extensions {[}Internet{]}. The R Foundation;
2020. Available:
\url{https://cran.r-project.org/doc/manuals/R-exts.html}

\leavevmode\hypertarget{ref-peng2011}{}%
17. Peng RD. Reproducible research in computational science. Science.
American Association for the Advancement of Science; 2011;334:
1226--1227.

\leavevmode\hypertarget{ref-goodman2016}{}%
18. Goodman SN, Fanelli D, Ioannidis JP. What does research
reproducibility mean? Science translational medicine. American
Association for the Advancement of Science; 2016;8: 341ps12--341ps12.

\leavevmode\hypertarget{ref-donoho2017}{}%
19. Donoho D. 50 years of data science. Journal of Computational and
Graphical Statistics. Taylor \& Francis; 2017;26: 745--766.

\leavevmode\hypertarget{ref-gentleman2004}{}%
20. Gentleman RC, Carey VJ, Bates DM, Bolstad B, Dettling M, Dudoit S,
et al. Bioconductor: Open software development for computational biology
and bioinformatics. Genome biology. Springer; 2004;5: R80.

\leavevmode\hypertarget{ref-holmes2018}{}%
21. Holmes S, Huber W. Modern statistics for modern biology
{[}Internet{]}. Cambridge University Press; 2018. Available:
\url{https://web.stanford.edu/class/bios221/book/index.html}

\leavevmode\hypertarget{ref-robinson2017}{}%
22. Robinson D. The impressive growth of r {[}Internet{]}. Stack
Overflow; 2017. Available:
\url{https://stackoverflow.blog/2017/10/10/impressive-growth-r/}

\nolinenumbers


\end{document}

