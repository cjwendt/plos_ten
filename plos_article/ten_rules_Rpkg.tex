% Template for PLoS
% Version 3.5 March 2018
%
% % % % % % % % % % % % % % % % % % % % % %
%
% -- IMPORTANT NOTE
%
% This template contains comments intended
% to minimize problems and delays during our production
% process. Please follow the template instructions
% whenever possible.
%
% % % % % % % % % % % % % % % % % % % % % % %
%
% Once your paper is accepted for publication,
% PLEASE REMOVE ALL TRACKED CHANGES in this file
% and leave only the final text of your manuscript.
% PLOS recommends the use of latexdiff to track changes during review, as this will help to maintain a clean tex file.
% Visit https://www.ctan.org/pkg/latexdiff?lang=en for info or contact us at latex@plos.org.
%
%
% There are no restrictions on package use within the LaTeX files except that
% no packages listed in the template may be deleted.
%
% Please do not include colors or graphics in the text.
%
% The manuscript LaTeX source should be contained within a single file (do not use \input, \externaldocument, or similar commands).
%
% % % % % % % % % % % % % % % % % % % % % % %
%
% -- FIGURES AND TABLES
%
% Please include tables/figure captions directly after the paragraph where they are first cited in the text.
%
% DO NOT INCLUDE GRAPHICS IN YOUR MANUSCRIPT
% - Figures should be uploaded separately from your manuscript file.
% - Figures generated using LaTeX should be extracted and removed from the PDF before submission.
% - Figures containing multiple panels/subfigures must be combined into one image file before submission.
% For figure citations, please use "Fig" instead of "Figure".
% See http://journals.plos.org/plosone/s/figures for PLOS figure guidelines.
%
% Tables should be cell-based and may not contain:
% - spacing/line breaks within cells to alter layout or alignment
% - do not nest tabular environments (no tabular environments within tabular environments)
% - no graphics or colored text (cell background color/shading OK)
% See http://journals.plos.org/plosone/s/tables for table guidelines.
%
% For tables that exceed the width of the text column, use the adjustwidth environment as illustrated in the example table in text below.
%
% % % % % % % % % % % % % % % % % % % % % % % %
%
% -- EQUATIONS, MATH SYMBOLS, SUBSCRIPTS, AND SUPERSCRIPTS
%
% IMPORTANT
% Below are a few tips to help format your equations and other special characters according to our specifications. For more tips to help reduce the possibility of formatting errors during conversion, please see our LaTeX guidelines at http://journals.plos.org/plosone/s/latex
%
% For inline equations, please be sure to include all portions of an equation in the math environment.
%
% Do not include text that is not math in the math environment.
%
% Please add line breaks to long display equations when possible in order to fit size of the column.
%
% For inline equations, please do not include punctuation (commas, etc) within the math environment unless this is part of the equation.
%
% When adding superscript or subscripts outside of brackets/braces, please group using {}.
%
% Do not use \cal for caligraphic font.  Instead, use \mathcal{}
%
% % % % % % % % % % % % % % % % % % % % % % % %
%
% Please contact latex@plos.org with any questions.
%
% % % % % % % % % % % % % % % % % % % % % % % %

\documentclass[10pt,letterpaper]{article}
\usepackage[top=0.85in,left=2.75in,footskip=0.75in]{geometry}

% amsmath and amssymb packages, useful for mathematical formulas and symbols
\usepackage{amsmath,amssymb}

% Use adjustwidth environment to exceed column width (see example table in text)
\usepackage{changepage}

% Use Unicode characters when possible
\usepackage[utf8x]{inputenc}

% textcomp package and marvosym package for additional characters
\usepackage{textcomp,marvosym}

% cite package, to clean up citations in the main text. Do not remove.
% \usepackage{cite}

% Use nameref to cite supporting information files (see Supporting Information section for more info)
\usepackage{nameref,hyperref}

% line numbers
\usepackage[right]{lineno}

% ligatures disabled
\usepackage{microtype}
\DisableLigatures[f]{encoding = *, family = * }

% color can be used to apply background shading to table cells only
\usepackage[table]{xcolor}

% array package and thick rules for tables
\usepackage{array}

% create "+" rule type for thick vertical lines
\newcolumntype{+}{!{\vrule width 2pt}}

% create \thickcline for thick horizontal lines of variable length
\newlength\savedwidth
\newcommand\thickcline[1]{%
  \noalign{\global\savedwidth\arrayrulewidth\global\arrayrulewidth 2pt}%
  \cline{#1}%
  \noalign{\vskip\arrayrulewidth}%
  \noalign{\global\arrayrulewidth\savedwidth}%
}

% \thickhline command for thick horizontal lines that span the table
\newcommand\thickhline{\noalign{\global\savedwidth\arrayrulewidth\global\arrayrulewidth 2pt}%
\hline
\noalign{\global\arrayrulewidth\savedwidth}}


% Remove comment for double spacing
%\usepackage{setspace}
%\doublespacing

% Text layout
\raggedright
\setlength{\parindent}{0.5cm}
\textwidth 5.25in
\textheight 8.75in

% Bold the 'Figure #' in the caption and separate it from the title/caption with a period
% Captions will be left justified
\usepackage[aboveskip=1pt,labelfont=bf,labelsep=period,justification=raggedright,singlelinecheck=off]{caption}
\renewcommand{\figurename}{Fig}

% Use the PLoS provided BiBTeX style
% \bibliographystyle{plos2015}

% Remove brackets from numbering in List of References
\makeatletter
\renewcommand{\@biblabel}[1]{\quad#1.}
\makeatother



% Header and Footer with logo
\usepackage{lastpage,fancyhdr,graphicx}
\usepackage{epstopdf}
%\pagestyle{myheadings}
\pagestyle{fancy}
\fancyhf{}
%\setlength{\headheight}{27.023pt}
%\lhead{\includegraphics[width=2.0in]{PLOS-submission.eps}}
\rfoot{\thepage/\pageref{LastPage}}
\renewcommand{\headrulewidth}{0pt}
\renewcommand{\footrule}{\hrule height 2pt \vspace{2mm}}
\fancyheadoffset[L]{2.25in}
\fancyfootoffset[L]{2.25in}
\lfoot{\today}

%% Include all macros below

\newcommand{\lorem}{{\bf LOREM}}
\newcommand{\ipsum}{{\bf IPSUM}}

\usepackage{color}
\usepackage{fancyvrb}
\newcommand{\VerbBar}{|}
\newcommand{\VERB}{\Verb[commandchars=\\\{\}]}
\DefineVerbatimEnvironment{Highlighting}{Verbatim}{commandchars=\\\{\}}
% Add ',fontsize=\small' for more characters per line
\usepackage{framed}
\definecolor{shadecolor}{RGB}{248,248,248}
\newenvironment{Shaded}{\begin{snugshade}}{\end{snugshade}}
\newcommand{\AlertTok}[1]{\textcolor[rgb]{0.94,0.16,0.16}{#1}}
\newcommand{\AnnotationTok}[1]{\textcolor[rgb]{0.56,0.35,0.01}{\textbf{\textit{#1}}}}
\newcommand{\AttributeTok}[1]{\textcolor[rgb]{0.77,0.63,0.00}{#1}}
\newcommand{\BaseNTok}[1]{\textcolor[rgb]{0.00,0.00,0.81}{#1}}
\newcommand{\BuiltInTok}[1]{#1}
\newcommand{\CharTok}[1]{\textcolor[rgb]{0.31,0.60,0.02}{#1}}
\newcommand{\CommentTok}[1]{\textcolor[rgb]{0.56,0.35,0.01}{\textit{#1}}}
\newcommand{\CommentVarTok}[1]{\textcolor[rgb]{0.56,0.35,0.01}{\textbf{\textit{#1}}}}
\newcommand{\ConstantTok}[1]{\textcolor[rgb]{0.00,0.00,0.00}{#1}}
\newcommand{\ControlFlowTok}[1]{\textcolor[rgb]{0.13,0.29,0.53}{\textbf{#1}}}
\newcommand{\DataTypeTok}[1]{\textcolor[rgb]{0.13,0.29,0.53}{#1}}
\newcommand{\DecValTok}[1]{\textcolor[rgb]{0.00,0.00,0.81}{#1}}
\newcommand{\DocumentationTok}[1]{\textcolor[rgb]{0.56,0.35,0.01}{\textbf{\textit{#1}}}}
\newcommand{\ErrorTok}[1]{\textcolor[rgb]{0.64,0.00,0.00}{\textbf{#1}}}
\newcommand{\ExtensionTok}[1]{#1}
\newcommand{\FloatTok}[1]{\textcolor[rgb]{0.00,0.00,0.81}{#1}}
\newcommand{\FunctionTok}[1]{\textcolor[rgb]{0.00,0.00,0.00}{#1}}
\newcommand{\ImportTok}[1]{#1}
\newcommand{\InformationTok}[1]{\textcolor[rgb]{0.56,0.35,0.01}{\textbf{\textit{#1}}}}
\newcommand{\KeywordTok}[1]{\textcolor[rgb]{0.13,0.29,0.53}{\textbf{#1}}}
\newcommand{\NormalTok}[1]{#1}
\newcommand{\OperatorTok}[1]{\textcolor[rgb]{0.81,0.36,0.00}{\textbf{#1}}}
\newcommand{\OtherTok}[1]{\textcolor[rgb]{0.56,0.35,0.01}{#1}}
\newcommand{\PreprocessorTok}[1]{\textcolor[rgb]{0.56,0.35,0.01}{\textit{#1}}}
\newcommand{\RegionMarkerTok}[1]{#1}
\newcommand{\SpecialCharTok}[1]{\textcolor[rgb]{0.00,0.00,0.00}{#1}}
\newcommand{\SpecialStringTok}[1]{\textcolor[rgb]{0.31,0.60,0.02}{#1}}
\newcommand{\StringTok}[1]{\textcolor[rgb]{0.31,0.60,0.02}{#1}}
\newcommand{\VariableTok}[1]{\textcolor[rgb]{0.00,0.00,0.00}{#1}}
\newcommand{\VerbatimStringTok}[1]{\textcolor[rgb]{0.31,0.60,0.02}{#1}}
\newcommand{\WarningTok}[1]{\textcolor[rgb]{0.56,0.35,0.01}{\textbf{\textit{#1}}}}


\usepackage{booktabs}
\usepackage{longtable}
\usepackage{array}
\usepackage{multirow}
\usepackage{wrapfig}
\usepackage{float}
\usepackage{colortbl}
\usepackage{pdflscape}
\usepackage{tabu}
\usepackage{threeparttable}
\usepackage{threeparttablex}
\usepackage[normalem]{ulem}
\usepackage{makecell}
\usepackage{xcolor}
\usepackage{booktabs}
\usepackage{longtable}
\usepackage{array}
\usepackage{multirow}
\usepackage{wrapfig}
\usepackage{float}
\usepackage{colortbl}
\usepackage{pdflscape}
\usepackage{tabu}
\usepackage{threeparttable}
\usepackage{threeparttablex}
\usepackage[normalem]{ulem}
\usepackage{makecell}
\usepackage{xcolor}



\usepackage{forarray}
\usepackage{xstring}
\newcommand{\getIndex}[2]{
  \ForEach{,}{\IfEq{#1}{\thislevelitem}{\number\thislevelcount\ExitForEach}{}}{#2}
}

\setcounter{secnumdepth}{0}

\newcommand{\getAff}[1]{
  \getIndex{#1}{Stats,Math,ERHS}
}

\providecommand{\tightlist}{%
  \setlength{\itemsep}{0pt}\setlength{\parskip}{0pt}}

\begin{document}
\vspace*{0.2in}

% Title must be 250 characters or less.
\begin{flushleft}
{\Large
\textbf\newline{Ten simple rules for selecting an R package} % Please use "sentence case" for title and headings (capitalize only the first word in a title (or heading), the first word in a subtitle (or subheading), and any proper nouns).
}
\newline
% Insert author names, affiliations and corresponding author email (do not include titles, positions, or degrees).
\\
Caroline J. Wendt\textsuperscript{\getAff{Stats}, \getAff{Math}},
G. Brooke Anderson\textsuperscript{\getAff{ERHS}}\textsuperscript{*}\\
\bigskip
\textbf{\getAff{Stats}}Department of Statistics, Colorado State University, Fort Collins,
Colorado, United States of America\\
\textbf{\getAff{Math}}Department of Mathematics, Colorado State University, Fort Collins,
Colorado, United States of America\\
\textbf{\getAff{ERHS}}Department of Environmental \& Radiological Health Sciences, Colorado
State University, Fort Collins, Colorado, United States of America\\
\bigskip
* Corresponding author: Brooke.Anderson@colostate.edu\\
\end{flushleft}
% Please keep the abstract below 300 words
\section*{Abstract}
R is an increasingly preferred software environment for data analytics
and statistical computing among scientists and practitioners. Packages
markedly extend R's utility and ameliorate inefficient solutions. We
outline ten simple rules for finding relevant packages and determining
which package is best for your desired use.

% Please keep the Author Summary between 150 and 200 words
% Use first person. PLOS ONE authors please skip this step.
% Author Summary not valid for PLOS ONE submissions.
\section*{Author summary}
Write the author summary here. Do we want to include and author summary?

\linenumbers

% Use "Eq" instead of "Equation" for equation citations.
\emph{Text based on plos sample manuscript, see
\url{http://journals.plos.org/ploscompbiol/s/latex}}

\hypertarget{disclaimer}{%
\section{Disclaimer?}\label{disclaimer}}

Do we need to include a disclaimer in the margin like the one from
{[}1{]} that states: ``\textbf{Competing Interests}: The authors have no
affiliation with GitHub, nor with any other commercial entity mentioned
in this article. The views described here reflect their own views
without input from any third party organization.''

\begin{itemize}
\tightlist
\item
  RStudio
\item
  ROpenSci
\item
  GitHub
\end{itemize}

\hypertarget{funding-acknowledgment}{%
\section{Funding acknowledgment?}\label{funding-acknowledgment}}

Do we need to include a funding acknowledgment in the margin as in the
examples?

\hypertarget{introduction}{%
\section{Introduction}\label{introduction}}

R is a programming language and environment for statistical computing
and graphics that was developed by statisticians and is maintained by an
international group of core contributors {[}2{]}. Unlike several popular
proprietary languages such as MATLAB or SAS, R is freely available, open
source, and easily extensible; the user can access, alter, extend, and
share code for various applications. Accordingly, a vibrant community of
R users has emerged, many of whom develop R packages to complement and
extend the functionality of base R. There are numerous analogies in
computing between programming and culinary arts: recipe structures,
coding cookbooks, and the like. To conceptualize packages, imagine you
are the chef, R is the kitchen, and packages are the special gadgets
which allow you to cook and bake new recipes. R packages are coding
delectables that enable you to perform practical tasks and solve
problems with interesting techniques.

Are there R packages for wrangling and cleaning data frames, designing
interactive applications for data visualization, or performing
dimensionality reduction? Yes! How do you \emph{find} an R package that
will help you train regression and classification models, assess the
beta diversity of a population, or analyze gene expression microarray
data? This answer is not as simple; there are tens of thousands of R
packages. As a natural consequence of the open-source nature of R, there
is considerable variation in the quality of R packages and nontrivial
differences among those that provide similar tools. The advanced R
user---having developed an intuition for their workflow---may be
relatively confident when searching for and selecting packages. By
contrast, an obstacle that characterizes learning R at the outset is the
struggle to (1) find a package to accomplish a particular task or solve
a problem of interest and (2) choose the best package to perform that
task. Even so, some obscure and complicated recipes make it difficult
for an experienced chef to select the best tools.

In both coding and life, we endeavor to make choices that optimize
outcomes. Just as one may go about shopping for shoes, deciding which
graduate program to pursue, or conducting a literature review, there is
a science behind selection. We inform our decisions by assessing,
comparing, and filtering options based on indicators of quality such as
utility, association, and reputation. Likewise, choosing an R package
requires attending to similar details. We outline ten simple rules for
finding and selecting R packages, so that you will spend less time
searching for the right tools and more time coding delightful recipes.

\hypertarget{rule-1-consider-your-purpose}{%
\section{Rule 1: Consider your
purpose}\label{rule-1-consider-your-purpose}}

Usually, there are several ways to accomplish the same task or arrive at
the same solution while programming, albeit some ways are more elegant
and efficient than others. To optimize your workflow, consider your
purpose by first identifying your task and goal, and then defining the
scope of the task and steps to achieve it. For some tasks, coding your
own recipe with existing tools is practical, while other tasks benefit
from new tools.

If the scope of your task is simple or reasonable, given your knowledge
and skills, using an R package may not be appropriate. That is, some
instances do not warrant additional functions, datasets, or
documentation. There can be advantages of coding in base R to complete
your task or solve your problem. In particular, when you code from
scratch, you know precisely what you are running; thus, your script may
be easier to decipher and maintain over time. Conversely, packages
require you to rely on shared code with features or underlying processes
of which you may not be aware.

Packages are favorable in the same sense as kitchen tools: when a task
has a broad or complex scope beyond what you can (or desire) to attempt
from scratch. While there are ways to cook up an algorithm using for
loops and conditionals in base R, a relevant package may accomplish the
same goal in a more reproducible, efficient manner, with less code and
fewer bugs. The more reasonable it is for a given task to be abstracted
away from its context, the more plausible it is that someone has
generalized its themes, developed efficient algorithms, and organized
them in an intuitive way to share with other R users. Extensive tasks
justify sophisticated frameworks with several functions that form a
cohesive package. Many processes that involve data---although varying in
application---are ubiquitous. Data manipulation is one such common task
that has been streamlined by packages such as \texttt{dplyr} and
\texttt{tidyr} {[}3,4{]} (see Table 1). Nevertheless, there are indeed
packages for seemingly singular tasks, which you may favor over coding
from scratch. Some R packages are small and have a specific use
conducive to tasks that require highly specialized functions. For
example, there is a package for converting English letters to numbers as
on a telephone keypad {[}5{]}.

Which existing functionalities in base R could be improved in context of
your problem? Which new functionalities would you like to add to base R
to expand what you can do? If you define your purpose by making
observations about and considering limitations of your current toolbox
before you start searching for new tools, you will be more likely to
recognize what you do and do not need. Develop a list of domain-specific
keywords that relate what you are trying to do to how one may go about
doing it in order to narrow your search. Identify the type of inputs you
have and envision working with them; contemplate the desired outputs and
corresponding format. For instance, suppose you are using Bioconductor
packages in analyses and have data you want to visualize; you must
consider that the inputs will be of a certain class---namely, S4
objects---which impose restrictions when creating graphics {[}6{]}. The
data visualization package you use should address such limitations.

\hypertarget{rule-2-spend-time-searching-find-and-collect-options}{%
\section{Rule 2: Spend time searching; find and collect
options}\label{rule-2-spend-time-searching-find-and-collect-options}}

Those who have used R packages may know that, although leveraging
existing tools can be advantageous, the initial challenge of finding a
suitable package for a given task can obstruct potential benefits.
Relatedly, new R users who are unfamiliar with the structure and syntax
of the language may be hindered by the process of finding packages
because they do not know where to search, what to look for, or how to
sift through options. R packages are mentioned in a variety of places
online, in print, and elsewhere. You can discover new packages any time
you learn R-related topics, collaborate with other R users, or browse
the internet.

\textbf{Learn}

Packages are essential to venturing beyond base R and, thus, quickly
become an integral aspect of advancing your R skills. When learning how
to program in R, you are typically introduced to some of the most common
packages, which tend to have more general purposes (\textbf{Table 1}).
In addition, reputable online tutorials, courses, and books are helpful
resources for acquiring knowledge about packages that are versatile and
reliable---many of which are short, accessible, and either affordable or
offered at \href{https://committedtotape.shinyapps.io/freeR/}{no cost}
to the learner. We recommend online R programming courses such as those
through \href{https://www.coursera.org/learn/r-programming}{Coursera}
and \href{https://www.codecademy.com/learn/learn-r}{Codeacademy} for
interactive learning and R book series including the
\href{https://rstudio.com/resources/books/}{RStudio books} and Springer
titles for further reading.

\textbf{Collaborate}

An inclusive and collaborative community is an overlooked, yet integral,
aspect of a software's success {[}7{]}. A defining feature of R is the
enthusiasm of its users and developers alike. The R community has a
widespread internet presence across various platforms; however, members
are markedly active on Twitter, a place where R users seek help, share
ideas, and stay informed on \texttt{\#rstats} happenings, including
releases of new packages {[}8{]}. Beyond social media, featured pages
and R blogs serve as another informal, up-to-date, and more detailed
avenue for communicating and promoting R-related information
(\textbf{Table ?}). Notably, Joseph Rickert, Ambassador at Large for
RStudio, writes monthly posts on the
\href{https://rviews.rstudio.com/}{R Views} blog highlighting
exceptional new R packages. Rickert also features special articles about
recently released packages and lists of top packages within certain
categories, including Computational Methods, Data, Machine Learning,
Medicine, Science, Statistics, Time Series, Utilities, Visualization.

A developer of an R package may intend for it to be private (exclusively
for personal or professional use) or public (at no cost and available
for use by anyone) {[}9{]}. If your task is specific to a line of
research, consult colleagues to see if they have relevant (private) code
they would be willing to share. Alternatively, literature in your field
may either introduce R packages developed to solve a unique data science
problem or mention packages used during the research process. The former
may be published in the \emph{Journal of Statistical Software},
\emph{The R Journal}, or \emph{BMC Bioinformatics}, for example, and
search queries that include \texttt{"R\ package"} with domain keywords
will narrow results. The latter requires identifying authors who used R
in their analyses; useful packages may be mentioned in the Methods
and/or References sections. You can search for packages directly by name
in Google Scholar: the \texttt{Cited\ by} link displays the number of
times a package has been cited and connects to a page with those
publications. Lastly, you can attend conferences to learn about recent R
package developments and applications. There are two major annual R
conferences: \href{https://rstudio.com/conference/}{rstudio::conf} for
industry and \href{https://www.r-consortium.org/}{useR!} (not
exclusively) for academia. Conferences in your field may foster
connections with fellow scientists who use R for similar tasks and help
you collect information about packages related to your expertise. Talks
and presentations at conferences are often recorded and made available
online for playback.

\textbf{Browse}

The solution-seeking tactics we employ for many tasks nowadays may lead
you to think that finding R packages relies heavily on internet search
queries. Indeed, search engines such as Google return ample pages
related to anything \texttt{"\ldots{}in\ R"}. However, this approach can
lead to frustration and confusion when attempting to find a package
tailored to your purpose (see Rule 1). Instead, we recommend initially
searching for packages in repositories such as the Comprehensive R
Archive Network (CRAN), GitHub, or Bioconductor, all of which will be
further discussed in Rule 3. In particular,
\href{https://cran.r-project.org/web/views/}{CRAN Task Views} are
concentrated topics from certain disciplines and methodologies related
to statistical computing that categorize R packages by the tasks they
perform (e.g., Econometrics, Genetics, Optimization, Spatial). In the
HTML version, you can browse alphabetized subcategories within each Task
View and read concise descriptions to find tools with specific
functions. Alternatively, you can access Task Views directly from the R
console with \texttt{ctv::CRAN.views()}. To date, there are 41 Task
Views that collectively contain thousands of packages which are curated
and regularly tested. Moreover, CRAN Task Views provide tools that
enable you to automatically install all packages within a targeted area
of interest. Ultimately, by providing task-based organization, easy
simultaneous installation of related packages, meta-information, ensured
maintenance, and quality control, CRAN Task Views address several major
user-end issues that have arisen due to the sheer quantity of available
packages {[}10{]}. Relatedly,
\href{http://dirk.eddelbuettel.com/cranberries/index.html}{CRANberries}
is a hub of information about new, updated, and removed packages from
the CRAN network. Another place to find well-maintained tools is through
\href{https://ropensci.org/packages/}{rOpenSci packages}. These
filterable and searchable R packages are organized by name, maintainer,
description, and status (i.e., activity, association, review).

\hypertarget{rule-3-check-how-its-shared}{%
\section{Rule 3: Check how it's
shared}\label{rule-3-check-how-its-shared}}

Packages can be shared through a variety of platforms; a single package
is often available in multiple places. While repositories are the
primary way in which developers share packages for both public and
private use, there are alternatives. In lieu of making packages
accessible to everyone via internet repositories, some developers share
their code in zipped files or directly with collaborators. As far as R
packages are concerned, a repository is essentially a warehouse for
tools before they enter your kitchen; in computing terms, a repository
is analogous to a cloud because it is a central location in which data
is stored and managed. There is growing concern about whether there are
too many R packages and a simultaneous call for quality control
{[}11{]}. Some repositories impose vetting mechanisms that tame unwieldy
aspects of the R ecosystem by regularly checking underlying code and
managing corresponding webs of dependencies. The traditional
repositories for R packages are CRAN, Bioconductor, and GitHub; however,
there are lesser-known remote repositories that have unique properties.

Much of what we know about statistical methods and algorithms is wrapped
up in R packages---written and documented in various ways by R users.
The CRAN package repository is the most established and primary source
from which you can install R packages. You can simply use the
\texttt{install.packages()} function to install a package from CRAN;
source and/or binary code are automatically saved to your computer in a
designated package library {[}12{]}. When you want to use a particular
package, you load it to your R session via the \texttt{library()}
function from base R. Due to its longevity and historical role, Rickert
asserts that ``CRAN is the greatest open-source repository for
statistical computing knowledge in the world'' {[}9{]}. Evidently, much
of what we know about statistics is housed in CRAN where people share
such information. The R Foundation manages CRAN and imposes strict
regulatory practices for the selection and maintenance of the packages
they host. A package must pass a series of stability tests in accordance
with the CRAN Repository Policy before obtaining publication privileges
{[}13{]}. As such, CRAN only considers packages that make a substantial
contribution to statistical computing and graphics. CRAN maintainers
actively monitor source and contributed packages to ensure they are
compatible with the latest version of R and modify or remove packages
that do not uphold publication quality.

The Bioconductor project was motivated by a need for transparent,
reproducible, and efficient software in computational biology and
bioinformatics and supports the integration of computational rigor and
reproducibility in research on biological processes {[}14{]}. The R
environment and its package system is fundamental to the implementation
of Bioconductor's interoperable and object-oriented (S4) infrastructure.
Bioconductor software is in the form of coordinated, peer-reviewed R
packages. Bioconductor boasts a modularized design, wherein data
structures, functions, and the packages that contain them have distinct
roles that are accompanied by thorough documentation. Accessibility is a
pillar of the Bioconductor project, thus all forms of documentation,
including courses, vignettes, and interactive documents, are curated for
individuals with expertise in adjacent disciplines and minimal
experience with R (see Rule 4) {[}14{]}. Similar to CRAN, Bioconductor
has strict criteria for package submissions: a package must be relevant
to high-throughput genomic analysis, interoperable with other
Bioconductor packages, well-documented, supported in the long-term,
exclusive to Bioconductor, and comply with additional package guidelines
{[}15{]}. Bioconductor packages facilitate the analysis and
comprehension of biological data and help users solve problems that
arise when working with high-throughput genomic data such as those
related to microarrays, sequencing, flow cytometry, mass spectrometry,
and image analysis. As a subset of the R community, Bioconductor has a
supportive and innovative community that hosts annual meetings and
conferences.

The rapid uptick in package development and subsequent inter-repository
dependencies has sparked an ongoing debate on whether regulated
repositories such as CRAN and Bioconductor are preferable to other
distribution platforms, namely public version control systems like
GitHub {[}9,16{]}{[}17{]}. While there are practical downsides to their
restrictive practices, the benefits of exclusive repositories are
evident. Nevertheless, there are considerable advantages to hosting a
package on GitHub {[}9,16{]}. GitHub is a popular online user interface
and multi-purpose development platform that is also effective in
distributing R packages. An increasing number of packages are hosted on
GitHub during the development stages; if developers choose not to
distribute their package through GitHub, the stable release versions of
such packages are often published on CRAN or Bioconductor {[}18{]}.
GitHub provides R users with open access to package code, a timeline of
help resources (see Rule 4), a direct line of communication to
developers, and permits discovery of up-and-coming packages (see Rule
2). You can install the latest version of packages from GitHub via
\texttt{devtools::install\_github()}; however, the decentralized nature
of GitHub is not conducive to a tool that automatically locates and
installs corresponding dependencies {[}19{]}. For developers, GitHub
provides a convenient means by which anyone can share and contribute
public or private code without barriers to entry. Authors collaborate
within a version-controlled system to develop and distribute packages,
including those with dependencies that are not on CRAN or Bioconductor.
Further, the \texttt{drat} (Drat R Archive Template) package enables
developers to design individual repositories and suites of coordinated
repositories for packages that are stored in and/or distributed through
GitHub {[}20{]}. Both R users and package developers benefit from
interactive feedback channels through GitHub Issues and the Star rating
system.

Most novice R users will rarely encounter packages that are not shared
through the abovementioned platforms. The non-profit organization,
rOpenSci, runs a repository as part of their commitment to promote open
science practices through technical and social infrastructure for the R
community {[}21{]}. The repository only includes packages that have
passed their open review process, which is compliant with GitHub
infrastructure. Further, GitLab is git-based version control and
collaborative cloud for package production and deployment. It is an
alternative to GitHub for production of large-scale packages that
require continuous integration and continuous deployment for testing
data and code to ensure a stable end-product. There are platforms
strictly for the development, rather than distribution, of R packages
such as R-Forge and Omegahat, which are beyond the scope of this paper
{[}22{]}.

\hypertarget{rule-4-explore-the-availability-and-quality-of-help}{%
\section{Rule 4: Explore the availability and quality of
help}\label{rule-4-explore-the-availability-and-quality-of-help}}

There has been a call for the development of centralized resources in
statistical computing to enable a common understanding of software
quality and reliability: software information specified in publications,
domain-specific semantic dictionaries, and a single metadata resource
for statistical software {[}11{]}. No such resources have been
consolidated to serve these purposes and, given the decentralized nature
of today's information society, it is questionable whether they will
emerge. Current sources of information related to R packages are
dispersed and plentiful. On one hand, this allows users to explore
diverse solutions and discover new tools; on the other, not knowing
where to find help can lead to inefficient and ineffectual roundabouts.
Clearly, not all package resources share the same level of quality and
the fact that there are many resources in aggregate does not imply that
every package is associated with the same availability of resources.
While all R packages warrant some minimal standard of documentation,
beginners and users of complex packages might desire more.

You can access information about R packages along with an index of help
pages from the console via \texttt{help(package\ =\ "...")}. Package
information will vary; ideally, packages should have thorough
documentation, but at minimum, every R package should include a
\texttt{DESCRIPTION} file with metadata. The \texttt{DESCRIPTION} is a
succinct record of the package's purpose, dependencies, version, date,
license, associations, authors, and other technical details. The help
pages feature information about the structure of functions within the
package and contain executable examples to demonstrate the relationship
between various inputs and outputs. If the \texttt{DESCRIPTION} and help
pages alone leave you wanting, the package likely does not have further
(quality) documentation and therefore should not be your first choice,
if comparable options exist. In short, if the developer cannot initially
communicate how their tool works, then you may not want to use it in
your kitchen (read: if the instruction manual is useless, do not use the
blender).

Fortunately, plenty of R packages include additional documentation
beyond mere descriptions. The documentation that accompanies functions
within packages is critical; the fact that anyone can read the
documentation anytime and use it to guide their own work facilitates
extensibility. RDocumentation is a searchable
\href{https://www.rdocumentation.org/}{website}, package
(\texttt{install.packages("RDocumentation")}), and
\href{https://www.rdocumentation.org/docs/}{JSON API} for obtaining
integrated documentation for packages that are on CRAN, Bioconductor,
and GitHub. This is a subsidiary reason why packages shared on these
platforms tend to be superior (see Rule 3). RDocumentation may include:
an overview, installation instructions, examples of usage, functions,
guides, and vignettes. Most software documentation is rather technical
and extraneous to new users whereas a vignette is a practical type of
documentation in the form of a tutorial. A vignette is a detailed,
long-form document that describes the problems an R package can solve,
then illustrates applications through clear examples of code with
coordination of functions and explanations of outcomes. Packages can
have multiple vignettes; you can view or edit a specific vignette or
obtain a list of all vignettes for a package of interest via the
\texttt{vignette()} function.

Some packages are branded quite well and include a comprehensive set of
resources. Implicitly, this indicates that the authors are at least
serious about their package development, which may lead you to infer
that they know what they (and their package) are doing. Exemplary
documentation can signify an exceptional package. For instance, some
packages have websites and/or books. One popular method that developers
use to publish books about their package is through \texttt{bookdown}, a
relatively new extension of R Markdown that is structured in such a way
that integrates code, text, links, graphics, videos, and other content
in a format that can be published as a free, open, interactive, and
downloadable online book {[}23{]}. The \texttt{bookdown} package itself
has an \href{https://bookdown.org/yihui/bookdown/}{online book} that
details usage of the package {[}24{]}. \texttt{Rccp} is another package
with notable documentation and first-rate help resources; the developers
maintain both a \href{http://www.rcpp.org/}{main} and
\href{http://dirk.eddelbuettel.com/code/rcpp.html}{additional} website
with a wealth of organized information about the package and resources,
including examples, associations, publications, articles, blogs, code,
books, talks, a mailing list, and links to other resources with
\texttt{Rccp} tags. Aside from the documentation and resources from the
developer, further information about some R packages is available in
video tutorials, webinars, and code demonstrations (i.e., ``demos''). As
of RStudio v1.3, you can access tutorials powered by the \texttt{learnr}
package from the Tutorial pane in the IDE {[}25{]}. Finally, keep in
mind that RStudio creates cheatsheets of concise usage information for
popular packages through code and graphics organized by purpose.
Cheatsheets can be accessed directly via the RStudio Menu (Help
\textgreater{} Cheatsheets) or from the
\href{https://rstudio.com/resources/cheatsheets/}{RStudio website} on
which you can subscribe to cheatsheet updates and find translated
versions.

While using a package, anticipate complications beyond the scope of
documentation. In this case, you will use resources that involve
\emph{asking} for help---should the occasion arise, you want to be
assured that you will find a satisfactory answer. In the past, the
antiquated R-help mailing list was the only way to seek assistance;
since, the R community has formed, with inclusion and creative
problem-solving as hallmarks of its online presence {[}26{]}. The modern
R-help mailing list to which you can subscribe and send questions is
moderated by the R Core Development Team and includes additional facets
for major announcements about the development of R and availability of
new code (R-announce) and new or enhanced contributed packages
(R-packages) {[}27{]}. Certain packages have independent listservs;
\texttt{statnet} is an example of a suite of packages that has its own
\href{http://statnet.org/}{community listserv}. If a package has a
development repository on GitHub, check the Issues to verify that the
maintainer is responsive to posts and fixes bugs in a timely manner. In
addition, you can search discussion forums such as Stack Overflow, Cross
Validated, and Talk Stats to assess the activity associated with the
package in question. Analyses of the popularity of comparable data
analysis software in email and discussion traffic suggest that R is
rapidly becoming more prevalent and is the leading language by these
metrics {[}28,29{]}. When you encounter a problem, it is good practice
to first update the package to see if the problem is due to a bug in a
previous version---if the problem persists, seek help by finding or
posting a reproducible example {[}30{]}. Overall, avoid using a package
if the quality and quantity of related resources is lacking.

\hypertarget{rule-5-verify-the-credibility-of-the-authors}{%
\section{Rule 5: Verify the credibility of the
author(s)}\label{rule-5-verify-the-credibility-of-the-authors}}

Just as research is a library of shared insight, open source software is
a collection of shared tools. We care about who writes the articles we
read; we should also care, arguably more, about who creates the tools we
use. Although R is grounded in statistical computing and graphics, there
is variation R users' backgrounds and skills, and the same is true for R
developers. Associations and reputation are often a proxy for quality;
in this way, the process of evaluating and comparing R packages is no
different than other decisions. In fact, as you become more immersed in
the R community, you will find that name recognition is a crucial factor
that determines why you trust certain tools and hesitate to use others
{[}31{]}.

You can assess the credibility of R package developers through direct
and indirect signals. Who made the package? Consider whether expertise
in a certain domain is vital to the design and creation of the tool.
Research the authors' associations in academia, industry, and/or
laboratories and gauge the extent to which they have a primary role in R
development. Further, you can learn more about their experience, active
contributions to the R community, and history related to package
development by exploring their profiles on GitHub, Google Scholar,
Research Gate, Twitter, or personal or package websites. If an author
has such a history, peruse their portfolio of packages to see if any are
highly regarded or recognizable. Frequent commits and effective
resolutions of GitHub Issues can reveal the authors' priorities and
commitment. If the package was developed by multiple authors, research
each of them to evaluate the robustness of the team. By extension, these
indicators of developer involvement and reputation will help you discern
whether a package is worthy of your trust and time.

\hypertarget{rule-6-investigate-the-package-development}{%
\section{Rule 6: Investigate the package
development}\label{rule-6-investigate-the-package-development}}

You do not need to be a software engineer to identify strong package
development. Scientific software developers sometimes neglect best
practices; indeed, these shortcomings are evident in the tools they
create {[}32{]}. There are concrete ways to measure a tool's robustness
beyond whether it works for those who did not create it. R packages
often depend on other R packages; you should check the reputations of
such \emph{dependencies} when selecting a package---quality packages
will rely on a solid web of quality packages. What's more, like other
types of software, well-maintained R packages have multiple versions
corresponding to iterative releases to indicate that the package is
compatible with dependencies and loyally updated (e.g., bug fixes,
general improvements, new functionality) {[}1,12{]}. You can explore the
version history of a package to see if it is up-to-date. As a user,
there are two additional development protocols that you can further
investigate to assess the underlying stability and utility of a package:
unit tests and version control.

A responsible developer with a consistent and reproducible workflow will
implement formal testing on their code to examine expected behavior via
an automated process called unit testing {[}12,33{]}. Although
inconvenient at the outset, the developer---and by extension, the
package user---will benefit from unit testing, which results in fewer
bugs, a well-designed code structure, an efficient workflow, and robust
code that is not sensitive to major changes in the future {[}12{]}. To
alleviate the burdens of unit testing, \texttt{testthat} is a popular,
integrative R package that helps developers create reliable functions,
minimize error, and visualize progress through automatic code testing
{[}34{]}. Developers are also interested in quantifying the amount of
code in their package that has been tested. Test coverage, a measurement
of the proportion of code that has undergone unit testing, is an
objective metric for package developers, contributors, and users to
evaluate code quality. Many developers use the \texttt{covr} package to
generate reports and determine the magnitude of coverage on the
function, script, and package levels {[}35{]}. Relatedly, developers who
host their packages on GitHub, post status badges in the overview
(\texttt{README}) section of the repository webpage. GitHub badges are a
common self-imposed method to signal use of best practices and motivate
developers to produce a product that is high in quality and transparency
{[}36{]}. You may see, for example, license, dependency, or style
badges, all of which are good indicators of package caliber; however,
particular to this Rule, you should look for code coverage
(\texttt{codecov}) badges which reveal the percentage of test coverage.

As we mentioned in Rule 3, version control has an essential role in
package development and computational literacy more broadly {[}12,37{]}.
Version control is like a time capsule for your workflow because it
monitors and tracks changes to files as a project evolves, and stores
them as previous versions to be recovered if necessary. In other words,
``version control is as fundamental to programming as accurate notes
about lab procedures are to experimental science'' {[}37{]} p6. Git is a
decentralized open-source version control system that is useful
regardless of whether a project is independent or collaborative
{[}38{]}. GitHub works in conjunction with Git to provide a powerful
structured system to organize and manage components of a project for
others and your future self. A growing number of scientists have
research programs based in GitHub, which has become a revolutionary tool
for productive team science and distributed development efforts
{[}1,39{]}. As you may expect, Git coupled with GitHub is the version
control duo of choice among serious R package developers {[}12{]}. Thus,
if the package you are interested in using is among the thousands hosted
on GitHub, this is evidence that the developer is at least committed to
a logical, open, and reproducible workflow, suggestive of more time
spent designing their tool.

\hypertarget{rule-7-read-research-literature-seek-evidence-of-peer-review}{%
\section{Rule 7: Read, research literature, seek evidence of peer
review}\label{rule-7-read-research-literature-seek-evidence-of-peer-review}}

Peer review is an important aspect of scientific research, not least
because it establishes scholarly credibility. You can research
information about an R package in different forms of literature and
determine the extent to which it has been validated by the scientific
community. Some journals publish articles about R packages themselves
while others feature work that used a particular package (see Rule 2).
These packages are technically sound and have made a substantial
contribution to their fields and/or a common data science problem. In
response to the rising number of researchers creating tools and software
to work with their data, GitHub has granted developers the ability to
obtain a Digital Object Identifier (DOI) for any GitHub repository
archive so that code can be cited in academic literature {[}40{]}. If a
package has such a DOI, you can explore the network of research
associated with that package. Many R packages are associated with
content in books and series from scientific publishers such as Springer.
More directly, rOpenSci, is a unique example of an ecosystem of
open-source tools with peer reviewed R packages (see Rule 3) {[}21{]}.

\hypertarget{rule-8-quantify-how-established-the-package-is}{%
\section{Rule 8: Quantify how established the package
is}\label{rule-8-quantify-how-established-the-package-is}}

Consulting data to inform comparisons is never a bad idea; numerical
data associated with R packages will give you an impression of how
regarded the tool is and whether it has stood the test of time. Since
there are tens of thousands of R packages, you may be wondering how they
stack up in terms of popularity. On GitHub, a large number of Stars,
Forks, and Watchers associated with a package implies a substantial
following and widespread usage {[}31{]}. Likewise, the number of Google
Scholar citations is a metric of a package's impact on scientific
research and utility in research contexts (see Rule 2). RDocumentation
(see Rule 4) is rich with stats on R packages. RDocumentation hosts a
live \href{https://www.rdocumentation.org/trends}{Leaderboard} with
trends including the number of indexed packages and indexed functions,
most downloaded packages, most active maintainers, newest packages, and
newest updates. What's more, each package is assigned a percentile
rank---featured on its RDocumentation page---that quantifies the number
of times a package has been downloaded in a given month. A ranking
algorithm computes the direct, user-requested monthly downloads by
accounting for reverse dependencies (indirect downloads) so packages
that are commonly depended upon, and hence frequently downloaded, do not
skew the calculation {[}41{]}. You can research stats on corresponding
dependencies for a more holistic picture. To further determine if a
package is well-established in the R community, refer to the number of
versions and updates (more is better) as well as the date of the most
recent versions and updates (newer is better).

\hypertarget{rule-9-put-the-package-to-the-test}{%
\section{Rule 9: Put the package to the
test}\label{rule-9-put-the-package-to-the-test}}

If you are unable to decide whether to use a package based on prior
Rules, test it out. Similarly, if you have narrowed your options, work
with each to highlight differences. Exploring the package and engaging
in trial and error using your skills in context of your goal will
illuminate technical details and solidify any doubts. Note, in the case
that the package you want to try has been shared as a zipped file, you
can use a GitHub mirror of CRAN via
\texttt{devtools::install\_github("username/reponame")} as an
alternative to downloading a large or potentially corrupted zipped file.

At this point, what you have learned about the package should be quite
helpful. If the development and documentation are sound, the package
should come with a test script or working example that you can run after
installation {[}32{]}. Vignettes include many common data science
problems with solutions; you can run the code examples, tweak them, and
compare the outputs. In general, it is essential to know the behavior of
different functions within a package, how they interact, and how outputs
respond to changes in inputs. Suppose you are testing a package with
sparse documentation such that function descriptions often include
``\texttt{\ldots{}}'' and the argument descriptions seem incomplete.
This will be problematic if making a reasonable change to an argument
results in an incomprehensible error for which you cannot find help.
When this happens, you may not want to use the package for your task.

Sometimes packages do not interact well with other packages; a recipe
prepared with an odd combination of tools will not turn out. If you are
interested in working with a certain package but are already using other
packages in your workflow, you will need to verify that they work
together. More precisely, you should check the \emph{interoperability}
of all the packages you want to use. A given package may be highly
specialized and incompatible with certain packages in general, or simply
have a few tolerable quirks for which you can develop workarounds. There
are some packages that are masterful at doing what they are made to do,
yet incongruous with other packages. Such packages might, for example,
use S3 or S4 objects, which are two main approaches developers use to
implement object-oriented programming in R. Many packages for spatial
analysis as well as those from Bioconductor tend to use S4 objects to
represent data {[}14{]}. On the other hand, the \texttt{tidyverse}, a
unified suite of packages with an grammatical structure employed within
a ``pipeline'', expects data frame objects {[}42{]}. Thus, when you are
working in the \texttt{tidyverse}, you cannot incorporate S3 and S4
objects into the framework unless their corresponding functions are the
final step in the pipeline. The \texttt{broom} package, and the
bioinformatics analog, \texttt{biobroom}, aim to alleviate these
disruptions by converting untidy objects into tidy data, thereby making
it easier to integrate statistical functions into the structure of the
\texttt{tidyverse} workflow {[}43,44{]}. Furthermore, the \texttt{caret}
package facilitates interoperability for machine learning packages by
providing a uniform interface for modeling with various algorithms from
different packages that would otherwise have independent syntax
{[}45{]}.

\hypertarget{rule-10-develop-your-own-package}{%
\section{Rule 10: Develop your own
package}\label{rule-10-develop-your-own-package}}

Alternative solutions can be sought when a package to solve your data
science problem is nonexistent. An R package is the fundamental unit of
shareable code; rather than exclusively being a user of packages, you
can create them---more easily than you may think {[}12{]}. The reasons
why you might want to create a package are abundant, including
necessity, innovation, standardization, automation, specialty,
containment, organization, sharing, collaboration, and extensibility.
The essence of an R package is a self-contained piece of statistical
knowledge that can be used in combination with other self-contained
pieces of statistical knowledge of different shapes and sizes; the
uniquely structured functions within a package help us implement that
knowledge and weave it into novel scientific work.

Whatever your motivation, packages are simply tools; you can create a
package out of any collection of specialty functions. Packages need not
be formal nor entirely cohesive. For instance, personal R packages such
as \texttt{Hmisc} and \texttt{broman}, are comprised of miscellaneous
functions which the creator has developed and frequently uses
{[}46,47{]}. Functions are necessary for efficiency and warranted when
you repetitiously copy and paste your code while making slight
modifications after each iteration {[}30{]}. The concept of personal R
packages demonstrates a unique purpose for packages beyond the
conventional. R packages are not solely reserved for specific tasks with
comprehensive methods; rather, package development can help you learn
how to apply proper coding techniques to writing functions and
documentation with reproducibility and collaboration in mind {[}48{]}.

Although you may not anticipate that anyone else will use your tools,
following best practices for package development will yield more
favorable outcomes. As a consumer of shared packages, you know the
inherent benefits of robust software development relative to the quality
of code, data, documentation, versions, and tests {[}32{]}. Similarly,
creating a valuable package for personal use requires consideration for
your future self and anticipation of distributing your code, should the
need arise. Use version control and take advantage of existing
resources. Indeed, there are R packages that aid in package development
(e.g., \texttt{devtools}, \texttt{usethis}, \texttt{testthat},
\texttt{roxygen2}, \texttt{rlang}, \texttt{drat})
{[}19,49{]}{[}50{]}{[}51{]}{[}52{]}{[}20{]}. In the case of
collaboration, the R project within RStudio IDE, is compatible with
distributed development---a feature that couples well with version
control. There is no lack of effective organizational frameworks to
reference in the open-source R community; in fact, repositories for many
exemplary packages are available on GitHub. We recommend consulting
resources authored by expert R developers including
\href{https://r-pkgs.org/}{\emph{R Packages}} by Hadley Wickham and the
official manual,
\href{https://cran.r-project.org/doc/manuals/r-release/R-exts.html}{\emph{Writing
R Extensions}}, from CRAN {[}12,53{]}.

\hypertarget{conclusion}{%
\section{Conclusion}\label{conclusion}}

Computational reproducibility is surfacing as a central axiom in
academia, as researchers identify the need for means by which they can
implement transparent systems {[}54,55{]}. It follows that former
approaches and traditional methods tend to be at odds with productivity
and collaboration; some variability in scientific outcomes can be
attributed to differences in workflow thus the absence of automation is
deemed irresponsible {[}56{]}. The open source R language has become the
dominant quantitative programming environment in academic statistics,
enabling researchers to share workflows and re-execute scripts within
and across subsets of the scientific community {[}56{]}. R is
increasingly used by researchers in computational biology and
bioinformatics, one of many disciplines that is generating extensive
heterogenous and complex data that demand standard tools and rigorous
methods to support reproducibility {[}14,57{]}. More broadly, as the R
ecosystem---in which the life of modern data analysis thrives---rapidly
evolves alongside the burgeoning R community, R is exhibiting sustained
growth when compared to similar languages, particularly in academia,
healthcare, and government {[}28{]}.

R packages are a defining feature of the language insofar as many are
robust and user-friendly. Some of the most prominent R packages are a
result of the developer abstracting common elements of a data science
problem into a workflow that can be shared and accompanied by thorough
descriptions of the process and purpose. In this way, R packages have
effectively transformed how we interact with data in the modern day in,
perhaps, a more impactful manner than several revered contributions to
theoretical statistics {[}56{]}. Packages greatly enhance the user
experience and enable you to be more efficient and effective at learning
from data, regardless of prior experience. Nonetheless, the sheer
quantity and potential complexity of available R packages can undermine
their collective benefits. Finding and choosing packages, particularly
for beginners, can be daunting and difficult. R users often struggle to
sift through the tools at their disposal and wonder how to distinguish
appropriate usage. These ten simple rules for navigating the shared code
in the R community are intended to serve as a valuable page in your
computing cookbook---one that will evolve into intuition and yet remain
a reliable reference. May searching for and selecting proper tools no
longer spoil your appetite and dissuade you from discovering, trying,
creating, and sharing new recipes.

\hypertarget{table-1-general-packages}{%
\section{Table 1 (general packages)}\label{table-1-general-packages}}

\begin{Shaded}
\begin{Highlighting}[]
\KeywordTok{library}\NormalTok{(kableExtra)}
\KeywordTok{library}\NormalTok{(knitr)}
\end{Highlighting}
\end{Shaded}

\begin{Shaded}
\begin{Highlighting}[]
\CommentTok{# general packages data}
\NormalTok{gen_pkgs <-}\StringTok{ }\KeywordTok{data.frame}\NormalTok{(}
  \DataTypeTok{Package =} \KeywordTok{c}\NormalTok{(}\StringTok{"readr[note]"}\NormalTok{, }
              \StringTok{"dplyr[note]"}\NormalTok{,}
              \StringTok{"tidyr"}\NormalTok{,}
              
              \StringTok{"broom[note]"}\NormalTok{,}
              \StringTok{"purrr[note]"}\NormalTok{,}
              \StringTok{"caret"}\NormalTok{, }
              \StringTok{"keras"}\NormalTok{, }
              
              \StringTok{"ggplot2[note]"}\NormalTok{, }
              \StringTok{"kableExtra"}\NormalTok{, }
              \StringTok{"rmarkdown"}\NormalTok{),}
  
  \DataTypeTok{Description =} \KeywordTok{c}\NormalTok{(}\StringTok{"read rectangular data (e.g., csv, tsv, and fwf)"}\NormalTok{, }
                  \StringTok{"grammar of data manipulation"}\NormalTok{,}
                  \StringTok{"create tidy data"}\NormalTok{,}
                  
                  \StringTok{"tidy model output"}\NormalTok{, }
                  \StringTok{"functional programming tools"}\NormalTok{,}
                  \StringTok{"train classification and regression models"}\NormalTok{, }
                  \StringTok{"R interface to a neural network library"}\NormalTok{, }
                  
                  \StringTok{"data visualization"}\NormalTok{, }
                  \StringTok{"tables"}\NormalTok{,}
                  \StringTok{"reports"}\NormalTok{),}
  
  \DataTypeTok{Year =} \KeywordTok{c}\NormalTok{(}\StringTok{"readr"}\NormalTok{,}
           \StringTok{"dplyr"}\NormalTok{,}
           \StringTok{"tidyr"}\NormalTok{,}
  
           \StringTok{"broom"}\NormalTok{,}
           \StringTok{"purrr"}\NormalTok{,}
           \StringTok{"caret"}\NormalTok{, }
           \StringTok{"keras"}\NormalTok{,}
                    
           \StringTok{"ggplot2"}\NormalTok{,}
           \StringTok{"kableExtra"}\NormalTok{, }
           \StringTok{"rmarkdown"}\NormalTok{),}
  
  \DataTypeTok{Author =} \KeywordTok{c}\NormalTok{(}\StringTok{"readr Wickham et al."}\NormalTok{,}
             \StringTok{"dplyr Wickham et al."}\NormalTok{, }
             \StringTok{"tidyr Wickham et al."}\NormalTok{,}
  
             \StringTok{"broom Robinson et al."}\NormalTok{,}
             \StringTok{"purrr Henry et al."}\NormalTok{,}
             \StringTok{"caret Kuhn et al."}\NormalTok{, }
             \StringTok{"keras Falbel et al."}\NormalTok{,}
                    
             \StringTok{"ggplot2 Wickham et al."}\NormalTok{,}
             \StringTok{"kableExtra Zhu et al."}\NormalTok{, }
             \StringTok{"rmarkdown Allaire et al."}\NormalTok{),}
  
  \DataTypeTok{Documentation =} \KeywordTok{c}\NormalTok{(}\StringTok{"readr"}\NormalTok{, }
                    \StringTok{"dplyr"}\NormalTok{, }
                    \StringTok{"tidyr"}\NormalTok{,}
  
                    \StringTok{"broom"}\NormalTok{,}
                    \StringTok{"purrr"}\NormalTok{,}
                    \StringTok{"https://topepo.github.io/caret/index.html"}\NormalTok{, }
                    \StringTok{"keras"}\NormalTok{,}
                    
                    \StringTok{"ggplot2"}\NormalTok{,}
                    \StringTok{"kableExtra"}\NormalTok{, }
                    \StringTok{"rmarkdown"}\NormalTok{)}
\NormalTok{)}
\end{Highlighting}
\end{Shaded}

\begin{Shaded}
\begin{Highlighting}[]
\CommentTok{# general packages table}
\KeywordTok{kable}\NormalTok{(gen_pkgs, }\DataTypeTok{format =} \StringTok{"latex"}\NormalTok{, }\DataTypeTok{booktabs =} \OtherTok{TRUE}\NormalTok{) }\OperatorTok
\StringTok{  }\CommentTok{# scale}
\StringTok{  }\KeywordTok{kable_styling}\NormalTok{(}\DataTypeTok{latex_options =} \StringTok{"scale_down"}\NormalTok{) }\OperatorTok
\StringTok{  }\CommentTok{# separate rows by category}
\StringTok{  }\KeywordTok{pack_rows}\NormalTok{(}\StringTok{"Data Manipulation"}\NormalTok{, }\DecValTok{1}\NormalTok{, }\DecValTok{3}\NormalTok{) }\OperatorTok\StringTok{ }
\StringTok{  }\KeywordTok{pack_rows}\NormalTok{(}\StringTok{"Statistical Modeling"}\NormalTok{, }\DecValTok{4}\NormalTok{, }\DecValTok{7}\NormalTok{) }\OperatorTok\StringTok{ }
\StringTok{  }\KeywordTok{pack_rows}\NormalTok{(}\StringTok{"Data Visualization"}\NormalTok{, }\DecValTok{8}\NormalTok{, }\DecValTok{10}\NormalTok{) }\OperatorTok
\StringTok{  }\CommentTok{# column wrap}
\StringTok{  }\KeywordTok{column_spec}\NormalTok{(}\DecValTok{1}\NormalTok{, }\DataTypeTok{width =} \StringTok{"10em"}\NormalTok{) }\OperatorTok\StringTok{ }
\StringTok{  }\KeywordTok{column_spec}\NormalTok{(}\DecValTok{2}\NormalTok{, }\DataTypeTok{width =} \StringTok{"20em"}\NormalTok{) }\OperatorTok\StringTok{ }
\StringTok{  }\CommentTok{# bold column names}
\StringTok{  }\KeywordTok{row_spec}\NormalTok{(}\DecValTok{0}\NormalTok{, }\DataTypeTok{bold =}\NormalTok{ T) }\OperatorTok\StringTok{ }
\StringTok{  }\KeywordTok{add_footnote}\NormalTok{(}\KeywordTok{c}\NormalTok{(}\StringTok{"See the tidyverse"}\NormalTok{,}
                 \StringTok{"See the tidyverse"}\NormalTok{,}
                 \StringTok{"See the biobroom analog in Bioconductor"}\NormalTok{,}
                 \StringTok{"See the tidyverse"}\NormalTok{,}
                 \StringTok{"See the tidyverse"}\NormalTok{),}
               \DataTypeTok{notation =} \StringTok{"symbol"}\NormalTok{)}
\end{Highlighting}
\end{Shaded}

\begin{table}[H]
\centering
\resizebox{\linewidth}{!}{
\begin{tabular}{>{\raggedright\arraybackslash}p{10em}>{\raggedright\arraybackslash}p{20em}lll}
\toprule
\textbf{Package} & \textbf{Description} & \textbf{Year} & \textbf{Author} & \textbf{Documentation}\\
\midrule
\addlinespace[0.3em]
\multicolumn{5}{l}{\textbf{Data Manipulation}}\\
\hspace{1em}readr\textsuperscript{*} & read rectangular data (e.g., csv, tsv, and fwf) & readr & readr Wickham et al. & readr\\
\hspace{1em}dplyr\textsuperscript{\dag} & grammar of data manipulation & dplyr & dplyr Wickham et al. & dplyr\\
\hspace{1em}tidyr & create tidy data & tidyr & tidyr Wickham et al. & tidyr\\
\addlinespace[0.3em]
\multicolumn{5}{l}{\textbf{Statistical Modeling}}\\
\hspace{1em}broom\textsuperscript{\ddag} & tidy model output & broom & broom Robinson et al. & broom\\
\hspace{1em}purrr\textsuperscript{\S} & functional programming tools & purrr & purrr Henry et al. & purrr\\
\hspace{1em}caret & train classification and regression models & caret & caret Kuhn et al. & https://topepo.github.io/caret/index.html\\
\hspace{1em}keras & R interface to a neural network library & keras & keras Falbel et al. & keras\\
\addlinespace[0.3em]
\multicolumn{5}{l}{\textbf{Data Visualization}}\\
\hspace{1em}ggplot2\textsuperscript{\P} & data visualization & ggplot2 & ggplot2 Wickham et al. & ggplot2\\
\hspace{1em}kableExtra & tables & kableExtra & kableExtra Zhu et al. & kableExtra\\
\hspace{1em}rmarkdown & reports & rmarkdown & rmarkdown Allaire et al. & rmarkdown\\
\bottomrule
\multicolumn{5}{l}{\textsuperscript{*} See the tidyverse}\\
\multicolumn{5}{l}{\textsuperscript{\dag} See the tidyverse}\\
\multicolumn{5}{l}{\textsuperscript{\ddag} See the biobroom analog in Bioconductor}\\
\multicolumn{5}{l}{\textsuperscript{\S} See the tidyverse}\\
\multicolumn{5}{l}{\textsuperscript{\P} See the tidyverse}\\
\end{tabular}}
\end{table}

\begin{Shaded}
\begin{Highlighting}[]
\CommentTok{## trying to separate color; striped by group}
\CommentTok{#  row_spec(1:3 - 1, extra_latex_after = "\textbackslash{}\textbackslash{}rowcolor\{gray!6\}")}
\CommentTok{#  row_spec(0:3, extra_latex_after = "\textbackslash{}\textbackslash{}rowcolor\{orange!6\}") %>% }
\CommentTok{#  row_spec(4:6, extra_latex_after = "\textbackslash{}\textbackslash{}rowcolor\{gray!6\}") %>% }
\CommentTok{#  row_spec(7:11, extra_latex_after = "\textbackslash{}\textbackslash{}rowcolor\{gray!6\}")}

\CommentTok{## QUESTIONS }
\CommentTok{# Code font for package names in " "? \textbackslash{}texttt\{\}?}
\CommentTok{# How do you repeat same symbol on multiple items with one footnote?}
\CommentTok{# How do you separate colors and stripe by group?}
\CommentTok{# Add title}
\CommentTok{# Add caption}
\CommentTok{# Cite packages in bib and add references in table?}
\CommentTok{# Embed url link to package documentation? Do we want to link cheatsheets?}
\CommentTok{# How do you add link/reference to Table 1 in text in the template?}
\CommentTok{# How do you hide code for table in knitted pdf...include=FALSE errors?}
\CommentTok{# Title for column 2: description/purpose/usage?}
\CommentTok{# Length of description/purpose/usage for each package?}
\end{Highlighting}
\end{Shaded}

\hypertarget{supporting-information}{%
\section{Supporting information}\label{supporting-information}}

Do we need to include any supporting information?

\hypertarget{acknowledgements}{%
\section{Acknowledgements}\label{acknowledgements}}

{[}Acknowledgement of people who have helped{]}

{[}Funding acknowledgement{]}

\hypertarget{references}{%
\section*{References}\label{references}}
\addcontentsline{toc}{section}{References}

\hypertarget{refs}{}
\leavevmode\hypertarget{ref-perez2016}{}%
1. Perez-Riverol Y, Gatto L, Wang R, Sachsenberg T, Uszkoreit J, Veiga
Leprevost F da, et al. Ten simple rules for taking advantage of git and
github. PLoS computational biology. Public Library of Science; 2016;12.

\leavevmode\hypertarget{ref-Rproject2020}{}%
2. Team RC. The r project for statistical computing {[}Internet{]}. The
R Foundation; 2020. Available: \url{https://www.r-project.org/}

\leavevmode\hypertarget{ref-dplyr}{}%
3. Wickham H, François R, Henry L, Müller K. Dplyr: A grammar of data
manipulation {[}Internet{]}. 2020. Available:
\url{https://CRAN.R-project.org/package=dplyr}

\leavevmode\hypertarget{ref-tidyr}{}%
4. Wickham H, Henry L. Tidyr: Tidy messy data {[}Internet{]}. 2020.
Available: \url{https://CRAN.R-project.org/package=tidyr}

\leavevmode\hypertarget{ref-phonenumber}{}%
5. Myles S. Phonenumber: Convert letters to numbers and back as on a
telephone keypad {[}Internet{]}. 2015. Available:
\url{https://CRAN.R-project.org/package=phonenumber}

\leavevmode\hypertarget{ref-bioCproject}{}%
6. Huber W, Carey VJ, Gentleman R, Anders S, Carlson M, Carvalho BS, et
al. Orchestrating high-throughput genomic analysis with Bioconductor.
Nature Methods. 2015;12: 115--121. Available:
\url{http://www.nature.com/nmeth/journal/v12/n2/full/nmeth.3252.html}

\leavevmode\hypertarget{ref-smith2017}{}%
7. Smith D. The r community is one of r's best features {[}Internet{]}.
Revolutions. Microsoft; 2017. Available:
\url{https://blog.revolutionanalytics.com/2017/06/r-community.html}

\leavevmode\hypertarget{ref-ellis2017}{}%
8. Ellis SE. Hey! You there! You are welcome here {[}Internet{]}.
rOpenSci. NumFOCUS; 2017. Available:
\url{https://ropensci.org/blog/2017/06/23/community/}

\leavevmode\hypertarget{ref-rickert2018}{}%
9. Rickert J. What makes a great r package? {[}Internet{]}. RStudio;
2018. Available:
\url{https://rstudio.com/resources/rstudioconf-2018/what-makes-a-great-r-package-joseph-rickert/}

\leavevmode\hypertarget{ref-zeileis2005}{}%
10. Zeileis A. CRAN task views. R News. 2005;5: 39--40.

\leavevmode\hypertarget{ref-hornik2012}{}%
11. Hornik K. Are there too many r packages? Austrian Journal of
Statistics. 2012;41: 59--66.

\leavevmode\hypertarget{ref-wickham2015}{}%
12. Wickham H. R packages: Organize, test, document, and share your
code. "O'Reilly Media, Inc."; 2015.

\leavevmode\hypertarget{ref-cranpolicy2020}{}%
13. CRAN repository policy {[}Internet{]}. The R Foundation; 2020.
Available:
\url{https://cran.r-project.org/web/packages/policies.html\#Submission}

\leavevmode\hypertarget{ref-gentleman2004}{}%
14. Gentleman RC, Carey VJ, Bates DM, Bolstad B, Dettling M, Dudoit S,
et al. Bioconductor: Open software development for computational biology
and bioinformatics. Genome biology. Springer; 2004;5: R80.

\leavevmode\hypertarget{ref-biocpkgsub2020}{}%
15. Package submission {[}Internet{]}. Bioconductor; 2020. Available:
\url{https://www.bioconductor.org/developers/package-submission/}

\leavevmode\hypertarget{ref-mcelreath2020}{}%
16. McElreath R. Statistical rethinking: A bayesian course with examples
in r and stan. CRC press; 2020.

\leavevmode\hypertarget{ref-decan2016}{}%
17. Decan A, Mens T, Claes M, Grosjean P. When github meets cran: An
analysis of inter-repository package dependency problems. 2016 ieee 23rd
international conference on software analysis, evolution, and
reengineering (saner). IEEE; 2016. pp. 493--504.

\leavevmode\hypertarget{ref-decan2015}{}%
18. Decan A, Mens T, Claes M, Grosjean P. On the development and
distribution of r packages: An empirical analysis of the r ecosystem.
Proceedings of the 2015 european conference on software architecture
workshops. 2015. pp. 1--6.

\leavevmode\hypertarget{ref-devtools}{}%
19. Wickham H, Hester J, Chang W. Devtools: Tools to make developing r
packages easier {[}Internet{]}. 2020. Available:
\url{https://CRAN.R-project.org/package=devtools}

\leavevmode\hypertarget{ref-drat}{}%
20. Carl Boettiger DE with contributions by, Fultz N, Gibb S, Gillespie
C, Górecki J, Jones M, et al. Drat: 'Drat' r archive template
{[}Internet{]}. 2020. Available:
\url{https://CRAN.R-project.org/package=drat}

\leavevmode\hypertarget{ref-ropensci2020}{}%
21. Transforming science through open data and software {[}Internet{]}.
rOpenSci; 2020. Available: \url{https://ropensci.org/}

\leavevmode\hypertarget{ref-theussl2009}{}%
22. Theußl S, Zeileis A. Collaborative software development using
r-forge. Special invited paper on" the future of r". The R Journal. The
R Foundation for Statistical Computing; 2009;1: 9--14.

\leavevmode\hypertarget{ref-bookdown}{}%
23. Xie Y. Bookdown: Authoring books and technical documents with r
markdown {[}Internet{]}. 2020. Available:
\url{https://CRAN.R-project.org/package=bookdown}

\leavevmode\hypertarget{ref-xie2016}{}%
24. Xie Y. Bookdown: Authoring books and technical documents with r
markdown. Chapman; Hall/CRC; 2016.

\leavevmode\hypertarget{ref-ushey2020}{}%
25. Ushey K. RStudio 1.3 preview: Integrated tutorials {[}Internet{]}.
RStudio; 2020. Available:
\url{https://blog.rstudio.com/2020/02/25/rstudio-1-3-integrated-tutorials/}

\leavevmode\hypertarget{ref-chase2020}{}%
26. Chase W. Dataviz and the 20th anniversary of r, an interview with
hadley wickham {[}Internet{]}. Medium; 2020. Available:
\url{https://medium.com/nightingale/dataviz-and-the-20th-anniversary-of-r-an-interview-with-hadley-wickham-ea245078fc8a}

\leavevmode\hypertarget{ref-Rmail2020}{}%
27. Team RC. Mailing lists {[}Internet{]}. The R Foundation; 2020.
Available: \url{https://www.r-project.org/mail.html}

\leavevmode\hypertarget{ref-robinson2017}{}%
28. Robinson D. The impressive growth of r {[}Internet{]}. Stack
Overflow; 2017. Available:
\url{https://stackoverflow.blog/2017/10/10/impressive-growth-r/}

\leavevmode\hypertarget{ref-muenchen2012}{}%
29. Muenchen RA. The popularity of data analysis software. URL
http://r4statscom/popularity. 2012;

\leavevmode\hypertarget{ref-wickham2014}{}%
30. Wickham H. Advanced r. CRC press; 2014.

\leavevmode\hypertarget{ref-leek2015}{}%
31. Leek J. How i decide when to trust an r package {[}Internet{]}.
2015. Available:
\url{https://simplystatistics.org/2015/11/06/how-i-decide-when-to-trust-an-r-package/}

\leavevmode\hypertarget{ref-taschuk2017}{}%
32. Taschuk M, Wilson G. Ten simple rules for making research software
more robust. PLoS computational biology. Public Library of Science;
2017;13.

\leavevmode\hypertarget{ref-hester2020}{}%
33. Hester J. How does covr work anyway? {[}Internet{]}. The R
Foundation; 2020. Available:
\url{https://cran.r-project.org/web/packages/covr/vignettes/how_it_works.html}

\leavevmode\hypertarget{ref-wickham2011}{}%
34. Wickham H. Testthat: Get started with testing. The R Journal.
2011;3: 5--10. Available:
\url{https://journal.r-project.org/archive/2011-1/RJournal_2011-1_Wickham.pdf}

\leavevmode\hypertarget{ref-covr}{}%
35. Hester J. Covr: Test coverage for packages {[}Internet{]}. 2020.
Available: \url{https://CRAN.R-project.org/package=covr}

\leavevmode\hypertarget{ref-barts2018}{}%
36. Barts C. How to use github badges to stop feeling like a noob
{[}Internet{]}. freeCodeCamp; 2018. Available:
\url{https://www.freecodecamp.org/news/how-to-use-badges-to-stop-feeling-like-a-noob-d4e6600d37d2/}

\leavevmode\hypertarget{ref-wilson2006}{}%
37. Wilson GV. Where's the real bottleneck in scientific computing?
American Scientist. 2006;94: 5.

\leavevmode\hypertarget{ref-bryan2018}{}%
38. Bryan J. Excuse me, do you have a moment to talk about version
control? The American Statistician. Taylor \& Francis; 2018;72: 20--27.

\leavevmode\hypertarget{ref-perkel2016}{}%
39. Perkel J. Democratic databases: Science on github. Nature News.
2016;538: 127.

\leavevmode\hypertarget{ref-smith2014}{}%
40. Smith A. Improving github for science {[}Internet{]}. GitHub, Inc.
2014. Available:
\url{https://github.blog/2014-05-14-improving-github-for-science/}

\leavevmode\hypertarget{ref-vannoorenberghe2017}{}%
41. Vannoorenberghe L. RDocumentation: Scoring and ranking
{[}Internet{]}. DataCamp; 2017. Available:
\url{https://www.datacamp.com/community/blog/rdocumentation-ranking-scoring}

\leavevmode\hypertarget{ref-tidyverse}{}%
42. Wickham H, Averick M, Bryan J, Chang W, McGowan LD, François R, et
al. Welcome to the tidyverse. Journal of Open Source Software. 2019;4:
1686.
doi:\href{https://doi.org/10.21105/joss.01686}{10.21105/joss.01686}

\leavevmode\hypertarget{ref-broom}{}%
43. Robinson D, Hayes A. Broom: Convert statistical analysis objects
into tidy tibbles {[}Internet{]}. 2020. Available:
\url{https://CRAN.R-project.org/package=broom}

\leavevmode\hypertarget{ref-biobroom}{}%
44. Andrew J. Bass SL David G. Robinson. Biobroom: Turn bioconductor
objects into tidy data frames {[}Internet{]}. 2020. Available:
\url{https://github.com/StoreyLab/biobroom}

\leavevmode\hypertarget{ref-caret}{}%
45. Kuhn M. Caret: Classification and regression training
{[}Internet{]}. 2020. Available:
\url{https://CRAN.R-project.org/package=caret}

\leavevmode\hypertarget{ref-Hmisc}{}%
46. Harrell Jr FE, Charles Dupont, others. Hmisc: Harrell miscellaneous
{[}Internet{]}. 2020. Available:
\url{https://CRAN.R-project.org/package=Hmisc}

\leavevmode\hypertarget{ref-broman}{}%
47. Broman KW. Broman: Karl broman's r code {[}Internet{]}. 2020.
Available: \url{https://CRAN.R-project.org/package=broman}

\leavevmode\hypertarget{ref-parker2013}{}%
48. Parker H. Personal r packages {[}Internet{]}. 2013. Available:
\url{https://hilaryparker.com/2013/04/03/personal-r-packages/}

\leavevmode\hypertarget{ref-usethis}{}%
49. Wickham H, Bryan J. Usethis: Automate package and project setup
{[}Internet{]}. 2019. Available:
\url{https://CRAN.R-project.org/package=usethis}

\leavevmode\hypertarget{ref-testthat}{}%
50. Wickham H. Testthat: Get started with testing. The R Journal.
2011;3: 5--10. Available:
\url{https://journal.r-project.org/archive/2011-1/RJournal_2011-1_Wickham.pdf}

\leavevmode\hypertarget{ref-roxygen2}{}%
51. Wickham H, Danenberg P, Csárdi G, Eugster M. Roxygen2: In-line
documentation for r {[}Internet{]}. 2020. Available:
\url{https://CRAN.R-project.org/package=roxygen2}

\leavevmode\hypertarget{ref-rlang}{}%
52. Henry L, Wickham H. Rlang: Functions for base types and core r and
'tidyverse' features {[}Internet{]}. 2020. Available:
\url{https://CRAN.R-project.org/package=rlang}

\leavevmode\hypertarget{ref-Rcore2020}{}%
53. Team RC. Writing r extensions {[}Internet{]}. The R Foundation;
2020. Available:
\url{https://cran.r-project.org/doc/manuals/R-exts.html}

\leavevmode\hypertarget{ref-peng2011}{}%
54. Peng RD. Reproducible research in computational science. Science.
American Association for the Advancement of Science; 2011;334:
1226--1227.

\leavevmode\hypertarget{ref-goodman2016}{}%
55. Goodman SN, Fanelli D, Ioannidis JP. What does research
reproducibility mean? Science translational medicine. American
Association for the Advancement of Science; 2016;8: 341ps12--341ps12.

\leavevmode\hypertarget{ref-donoho2017}{}%
56. Donoho D. 50 years of data science. Journal of Computational and
Graphical Statistics. Taylor \& Francis; 2017;26: 745--766.

\leavevmode\hypertarget{ref-holmes2018}{}%
57. Holmes S, Huber W. Modern statistics for modern biology
{[}Internet{]}. Cambridge University Press; 2018. Available:
\url{https://web.stanford.edu/class/bios221/book/index.html}

\nolinenumbers


\end{document}

